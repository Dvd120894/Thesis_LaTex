%%%%%%%%%%%%%%%%%%%%%%%%%%%%%%%%%%%%%%%%%
% Davide Stocco Thesis
% XeLaTeX Template
% Version 1.0 (23/09/19)
%
% This template has been downloaded from:
% http://www.LaTeXTemplates.com
%
% Original authors:
% Federico Maggi (fede@maggi.cc) with extensive modifications by:
% Vel (vel@latextemplates.com)
% Davide Stocco (davidestoc94@gmail.com)
%
% License:
% CC BY-NC-SA 3.0 (http://creativecommons.org/licenses/by-nc-sa/3.0/)
%
% Important notes:
% This template needs to be compiled with XeLaTeX.
%
% Most of the document content and packages are specified within structure.tex
% so if you need to make modifications to the template have a look there first!
%
% This template uses several fonts that are not available on most operating 
% systems by default. These are: Adobe Caslon Pro, Envy Code R and 
% Optima Regular. You will either need to obtain these and install them on your
% system or change them to different fonts. Simply go to the Fonts block just
% below here and modify their names to other fonts. You can also comment them 
% out completely to use the default LaTeX font.
%
%%%%%%%%%%%%%%%%%%%%%%%%%%%%%%%%%%%%%%%%%

%----------------------------------------------------------------------------------------
%	PACKAGES AND OTHER DOCUMENT CONFIGURATIONS
%----------------------------------------------------------------------------------------

\documentclass[12pt,a4paper,twoside]{memoir} % Change font size here (allowable values are 9pt-12pt), change the paper size, specify one or two sided printing and specify whether to show trimming lines

% Base line
\renewcommand{\baselinestretch}{1.2}

\input{structure.tex} % Include the file containing the code defining the structure and style of the document

%------------------------------------------------
% Thesis Information

\title{\textbf{Algorithm for Tire Contact Patch Evaluation in Soft Real Time}} % Thesis title

\author{Davide Stocco} % Author name

\date{Academic Year 2019 $\cdot$ 2020} % The date

\newcommand{\institution}{\large{University of Trento}\xspace} % University/institution name

\newcommand{\department}{\large{Department of Industrial Engineering}\xspace} % Department name

%------------------------------------------------
% Fonts

\usepackage{fontspec}
\defaultfontfeatures{Mapping=tex-text}
\setromanfont[Ligatures={Common}]{Adobe Caslon Pro} % Normal document font
%\setmonofont[Scale=1]{Courier} % Mono spaced font (\texttt{})
\setsansfont[Scale=0.9]{Optima} % Sans-serif font (\textsf{})

\renewcommand*{\acffont}[1]{{\normalsize\itshape #1}} % Font style for the acronym text (e.g. Do It Yourself)
\renewcommand*{\acfsfont}[1]{{\normalsize\upshape #1}} % Font style for the acronym in bracket (e.g. (DIY))

%------------------------------------------------
% Hyphenations

\hyphenation{a-no-ma-lous a-no-ma-ly amounts breaches} % Specify custom hyphenation points in words with dashes where you would like hyphenation to occur, or alternatively, don't put any dashes in a word to stop hyphenation altogether

%------------------------------------------------
% Bibliography


%----------------------------------------------------------------------------------------
%	TITLE PAGE
%----------------------------------------------------------------------------------------

\renewcommand{\maketitlehooka}{
\centering
\vspace{-2em}
\includegraphics[width=4cm]{Figures/unitn}\\[.5cm] % Institution logo
\institution\\[.1cm] % Print institution name
\emph{\department}\\[.2cm] % Print department name
MASTER'S DEGREE IN MECHATRONICS ENGINEERING % Degree or other information
\par
\hrulefill
\vfill
Master's Thesis
\vfill}
\renewcommand{\maketitlehookb}{\vfill}
\renewcommand{\maketitlehookc}{
\vfill
\begin{flushleft}
Supervisor:\\
\textbf{Prof. Enrico Bertolazzi}\\%[.3cm] % Supervisor's name
\end{flushleft}
\vfill}
\preauthor{\begin{flushright} Graduant:\\\bfseries} % Text prior to the author name - right aligned and bold
\postauthor{\end{flushright}} % After the author name, stop right alignment

%----------------------------------------------------------------------------------------

\makeindex % Write an index file

\begin{document}

\begin{titlingpage}
\maketitle % Print the title page
\end{titlingpage}

\frontmatter % Use roman page numbering style (i, ii, iii, iv...) for the pre-content pages

%----------------------------------------------------------------------------------------
%	PREFACE
%----------------------------------------------------------------------------------------

%\section*{Preface}
%This thesis embraces all the efforts that I put during the last three years as a PhD student at Politecnico di Milano. I have been working under the supervision of Prof. S. Zanero and Prof. G. Serazzi, who is also the leader of the research group I am part of. In this time frame I had the wonderful opportunity of being ``initiated'' to research, which radically changed the way I look at things: I found my natural \emph{``thinking outside the box''} attitude --- that was probably well-hidden under a thick layer of lack-of-opportunities, I took part of very interesting joint works --- among which the year I spent at the Computer Security Laboratory at UC Santa Barbara is at the first place, and I discovered the Zen of my life.
%
%\begin{flushright}
%\textsc{\theauthor}\\
%Trento\\
%January 2020
%\end{flushright}

\cleartoverso % Force a break to an even page

%----------------------------------------------------------------------------------------
%	ABSTRACT
%----------------------------------------------------------------------------------------

\begin{abstract}
This dissertation details \dots
\end{abstract}

\cleartoverso % Force a break to an even page

%----------------------------------------------------------------------------------------
%	TABLE OF CONTENTS
%----------------------------------------------------------------------------------------

\tableofcontents* % Print the table of contents

\cleartoverso % Force a break to an even page

%----------------------------------------------------------------------------------------
%	LIST OF FIGURES
%----------------------------------------------------------------------------------------

\listoffigures % Print the list of figures

\cleartoverso % Force a break to an even page

%----------------------------------------------------------------------------------------
%	LIST OF TABLES
%----------------------------------------------------------------------------------------

\listoftables % Print the list of tables

\cleartoverso % Force a break to an even page

%----------------------------------------------------------------------------------------
%	ACRONYMS
%----------------------------------------------------------------------------------------

% Dot fill spacing
\makeatletter
\renewcommand \dotfill {\leavevmode \cleaders \hb@xt@ .75em{\hss .\hss}\hfill \hspace{1em} \kern \z@}
\makeatother

\chapter*{Elenco degli acronimi}
\begin{acronym}\addtolength{\itemsep}{-\baselineskip}
	\acro{AABB}{Axis Aligned Bounding Box}
	\acro{ADAS}{Advanced Driver-Assistance Systems}
	\acro{ANSI}{American National Standard Institute}
	\acro{AOBB}{Arbitrarily Oriented Bounding Box}
	\acro{ASCII}{American Standard for Information Interxchange}
	\acro{BB}{Bounding Box}
	\acro{BVH}{Bounding Volume Hierarchy}
	\acro{CAD}{Computer-Aided Design}
	\acro{CAE}{Computer-Aided Engineering}
	\acro{CAGD}{Computer-Aided Geometric Design}
	\acro{CAM}{Computer-Aided Manufacturing}
	\acro{CdM}{Centro di Massa}
	\acro{ETRTO}{European Tyre and Rim Technical Organisation}
	\acro{GdL}{Gradi di Libertò}
	\acro{GIS}{Geographic Information Systems}
	\acro{HIL}{Hardware in the Loop}
	\acro{ISO}{International Organization for Standardization}
	\acro{MBB}{Minimum Bounding Box}
	\acro{OOBB}{Object Oriented Bounding Box}
	\acro{RDF}{Road Data File}
	\acro{SIL}{Software in the Loop}
\end{acronym} % Include a List of Acronyms section using acronyms.tex where they are defined

\cleartoverso % Force a break to an even page

%----------------------------------------------------------------------------------------
%	CONTENT CHAPTERS
%----------------------------------------------------------------------------------------

\mainmatter % Begin numeric (1,2,3...) page numbering

\chapterstyle{thesis} % Change the style of the Chapter header to that defined in structure.tex

\pagestyle{Ruled} % Include the chapter/section in the header along with a horizontal rule underneath

\chapter{Introduction}
\label{introduction}

\cite{Flocke} % Include the introduction chapter
\chapter{Computation Geometry Algorithms}
\label{Geom_Algos}
%
\section{Ray-Triangle Intersection Algorithm}
One of the many problems in Computer Graphics is the ray-triangle intersection.
%
\subsection{The M\"oller-Trumbore Algorithm}
%
The inputs of the  M\"oller-Trumbore algorithm are:
\begin{itemize}
	\item Triangle vertices $(V_1, V_2, V_3)$;
	\item Segment points $(Q_1, Q_2)$.
\end{itemize}
%
\noindent
\begin{minipage}[t]{.5\textwidth}
\raggedright
\textit{With Back-Face Culling}\\
\vspace{1em}
$Q = Q_2 - Q_1$\\
$E_1 = V_2 - V_1$\\
$E_2 = V_3 - V_1$\\
$A = Q \times E_2$\\
$D = A \cdot E_1$\\
$\mathbf{if} \, (D > \varepsilon) \{$\\
\quad $T = Q_1 - V_1$\\
\quad $u = A \cdot T$\\
\quad $\mathbf{if} \, (u < 0.0 \, || \, u > D) \{$\\
\quad \quad $\mathbf{return \, false}$\\
\quad $\}$\\
\quad $B = T \times E_1$\\
\quad $v = B \cdot Q$\\
\quad $\mathbf{if} \, (v < 0.0 \, || \, u + v > D) \{$\\
\quad \quad $\mathbf{return \, false}$\\
\quad $\}$\\
$\} \, \mathbf{else \, if} \, (D < -\varepsilon) \{$\\
\quad $T = Q_1 - V_1$\\
\quad $u = A \cdot T$\\
\quad $\mathbf{if} \, (u > 0.0 \, || \, u < D) \{$\\
\quad \quad $\mathbf{return \, false}$\\
\quad $\}$\\
\quad $B = T \times E_1$\\
\quad $v = B \cdot Q$\\
\quad $\mathbf{if} \, (v > 0.0 \, || \, u + v < D) \{$\\
\quad \quad $\mathbf{return \, false}$\\
\quad $\}$\\
$\} \, \mathbf{else} \, \{$\\
\quad $\mathbf{return \, false}$\\
$\}$\\
$D_{inv} = 1.0 / D$\\
$t = (B \cdot E_2) * D_{inv}$\\
$\mathbf{if} \, (t > 0.0) \{$\\
\quad $P = Q + D * t$\\
\quad $\mathbf{return \, true}$\\
$\} \, \mathbf{else} \, \{$\\
\quad $\mathbf{return \, false}$\\
$\}$
\end{minipage}% <---------------- Note the use of "%"
\begin{minipage}[t]{.5\textwidth}
\raggedright
\textit{Without Back-Face Culling}\\
\vspace{1em}
$Q = Q_2 - Q_1$\\
$E_1 = V_2 - V_1$\\
$E_2 = V_3 - V_1$\\
$A = Q \times E_2$\\
$D = A \cdot E_1$\\
$\mathbf{if} \, (D < \varepsilon) \{$\\
\quad $\mathbf{return \, false}$\\
$\}$\\
$T = Q_1 - V_1$\\
$u = A \cdot T$\\
$\mathbf{if} \, (u < 0.0 \, || \, u > D) \{$\\
\quad $\mathbf{return \, false}$\\
$\}$\\
$B = T \times E_1$\\
$v = B \cdot Q$\\
$\mathbf{if} \, (v < 0.0 \, || \, u + v > D) \{$\\
\quad \quad $\mathbf{return \, false}$\\
$\}$\\
$D_{inv} = 1.0 / D$\\
$t = (B \cdot E_2) * D_{inv}$\\
$\mathbf{if} \, (t > 0.0) \{$\\
\quad $P = Q + D * t$\\
\quad $\mathbf{return \, true}$\\
$\} \, \mathbf{else} \, \{$\\
\quad $\mathbf{return \, false}$\\
$\}$
\end{minipage} % Include the first content chapter
\include{Chapters/chapterExamples} % Include the first content chapter
%\include{Chapters/chapter2} % Include the second content chapter
%\include{Chapters/chapter3} % Include the third content chapter

\backmatter

\chapterstyle{default} % Reset the chapter style back to the default used for non-content chapters

%----------------------------------------------------------------------------------------
%	BIBLIOGRAPHY
%----------------------------------------------------------------------------------------

\printbibliography

%----------------------------------------------------------------------------------------
%	INDEX
%----------------------------------------------------------------------------------------

\printindex % Print the index

%----------------------------------------------------------------------------------------

\end{document}
