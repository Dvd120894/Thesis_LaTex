%%%%%%%%%%%%%%%%%%%%%%%%%%%%%%%%%%%%%%%%%
% Davide Stocco Thesis
% XeLaTeX Template
% Version 1.0 (23/09/19)
%
% This template has been downloaded from:
% http://www.LaTeXTemplates.com
%
% Original authors:
% Federico Maggi (fede@maggi.cc) with extensive modifications by:
% Vel (vel@latextemplates.com)
% Davide Stocco (davidestoc94@gmail.com)
%
% License:
% CC BY-NC-SA 3.0 (http://creativecommons.org/licenses/by-nc-sa/3.0/)
%
% Important notes:
% This template needs to be compiled with XeLaTeX.
%
% Most of the document content and packages are specified within structure.tex
% so if you need to make modifications to the template have a look there first!
%
% This template uses several fonts that are not available on most operating 
% systems by default. These are: Adobe Caslon Pro, Envy Code R and 
% Optima Regular. You will either need to obtain these and install them on your
% system or change them to different fonts. Simply go to the Fonts block just
% below here and modify their names to other fonts. You can also comment them 
% out completely to use the default LaTeX font.
%
%%%%%%%%%%%%%%%%%%%%%%%%%%%%%%%%%%%%%%%%%

%----------------------------------------------------------------------------------------
%	PACKAGES AND OTHER DOCUMENT CONFIGURATIONS
%----------------------------------------------------------------------------------------

\documentclass[12pt,a4paper,twoside]{memoir} % Change font size here (allowable values are 9pt-12pt), change the paper size, specify one or two sided printing and specify whether to show trimming lines

% Base line
\renewcommand{\baselinestretch}{1.25}
\usepackage[italian]{babel} 
\input{structure.tex} % Include the file containing the code defining the structure and style of the document

%------------------------------------------------
% Thesis Information

\title{\textbf{Algoritmi per la Valutazione del Contatto tra Pneumatico e Strada in Soft Real Time}} % Thesis title

\author{Davide Stocco} % Author name


\date{Anno Accademico 2019 $\cdot$ 2020} % The date

\newcommand{\institution}{\large{Universit\'a degli Studi di Trento}\xspace} % University/institution name

\newcommand{\department}{\large{Dipartimento di Ingegneria Industriale}\xspace} % Department name

%------------------------------------------------
% Fonts

\usepackage{fontspec}
\defaultfontfeatures{Mapping=tex-text}
\setromanfont[Ligatures={Common}]{Adobe Caslon Pro} % Normal document font
%\setmonofont[Scale=1]{Courier} % Mono spaced font (\texttt{})
\setsansfont[Scale=0.9]{Optima} % Sans-serif font (\textsf{})

\renewcommand*{\acffont}[1]{{\normalsize\itshape #1}} % Font style for the acronym text (e.g. Do It Yourself)
\renewcommand*{\acfsfont}[1]{{\normalsize\upshape #1}} % Font style for the acronym in bracket (e.g. (DIY))

%------------------------------------------------
% Hyphenations

\hyphenation{a-no-ma-lous a-no-ma-ly amounts breaches} % Specify custom hyphenation points in words with dashes where you would like hyphenation to occur, or alternatively, don't put any dashes in a word to stop hyphenation altogether

%------------------------------------------------
% Bibliography


%----------------------------------------------------------------------------------------
%	TITLE PAGE
%----------------------------------------------------------------------------------------

\renewcommand{\maketitlehooka}{
\centering
\vspace{-2em}
\includegraphics[width=4cm]{Figures/unitn}\\[.5cm] % Institution logo
\institution\\[.1cm] % Print institution name
\emph{\department}\\[.2cm] % Print department name
\scalebox{0.8}{CORSO DI LAUREA MAGISTRALE IN INGEGNERIA MECCATRONICA} % Degree or other information
\par
\hrulefill
\vfill
Tesi di Laurea
\vfill}
\renewcommand{\maketitlehookb}{\vfill}
\renewcommand{\maketitlehookc}{
\vfill
\begin{flushleft}
Relatore:\\
\textbf{Prof. Enrico Bertolazzi}\\%[.3cm] % Supervisor's name
\end{flushleft}
\vfill}
\preauthor{\begin{flushright} Laureando:\\\bfseries} % Text prior to the author name - right aligned and bold
\postauthor{\end{flushright}} % After the author name, stop right alignment

%----------------------------------------------------------------------------------------

\makeindex % Write an index file

\begin{document}

\begin{titlingpage}
\maketitle % Print the title page
\end{titlingpage}

\frontmatter % Use roman page numbering style (i, ii, iii, iv...) for the pre-content pages

%\cleartoverso % Force a break to an even page
\cleartooddpage % Force a break to an odd page
%----------------------------------------------------------------------------------------
%	ABSTRACT
%----------------------------------------------------------------------------------------

\begin{abstract}
This dissertation details \dots
\end{abstract}

%\cleartoverso % Force a break to an even page
\cleartooddpage % Force a break to an odd page

%----------------------------------------------------------------------------------------
%	TABLE OF CONTENTS
%----------------------------------------------------------------------------------------

\tableofcontents* % Print the table of contents

%\cleartoverso % Force a break to an even page
\cleartooddpage % Force a break to an odd page

%----------------------------------------------------------------------------------------
%	LIST OF FIGURES
%----------------------------------------------------------------------------------------

\listoffigures % Print the list of figures

%\cleartoverso % Force a break to an even page
\cleartooddpage % Force a break to an odd page

%----------------------------------------------------------------------------------------
%	LIST OF TABLES
%----------------------------------------------------------------------------------------

\listoftables % Print the list of tables

%\cleartoverso % Force a break to an even page
\cleartooddpage % Force a break to an odd page

%----------------------------------------------------------------------------------------
%	ACRONYMS
%----------------------------------------------------------------------------------------

% Dot fill spacing
\makeatletter
\renewcommand \dotfill {\leavevmode \cleaders \hb@xt@ .75em{\hss .\hss}\hfill \hspace{1em} \kern \z@}
\makeatother

\chapter*{Elenco degli acronimi}
\begin{acronym}\addtolength{\itemsep}{-\baselineskip}
	\acro{AABB}{Axis Aligned Bounding Box}
	\acro{ADAS}{Advanced Driver-Assistance Systems}
	\acro{ANSI}{American National Standard Institute}
	\acro{AOBB}{Arbitrarily Oriented Bounding Box}
	\acro{ASCII}{American Standard for Information Interxchange}
	\acro{BB}{Bounding Box}
	\acro{BVH}{Bounding Volume Hierarchy}
	\acro{CAD}{Computer-Aided Design}
	\acro{CAE}{Computer-Aided Engineering}
	\acro{CAGD}{Computer-Aided Geometric Design}
	\acro{CAM}{Computer-Aided Manufacturing}
	\acro{CdM}{Centro di Massa}
	\acro{ETRTO}{European Tyre and Rim Technical Organisation}
	\acro{GdL}{Gradi di Libertò}
	\acro{GIS}{Geographic Information Systems}
	\acro{HIL}{Hardware in the Loop}
	\acro{ISO}{International Organization for Standardization}
	\acro{MBB}{Minimum Bounding Box}
	\acro{OOBB}{Object Oriented Bounding Box}
	\acro{RDF}{Road Data File}
	\acro{SIL}{Software in the Loop}
\end{acronym} % Include a List of Acronyms section using acronyms.tex where they are defined

%\cleartoverso % Force a break to an even page
\cleartooddpage % Force a break to an odd page

%----------------------------------------------------------------------------------------
%	CONTENT CHAPTERS
%----------------------------------------------------------------------------------------

\mainmatter % Begin numeric (1,2,3...) page numbering

\chapterstyle{thesis} % Change the style of the Chapter header to that defined in structure.tex

\pagestyle{Ruled} % Include the chapter/section in the header along with a horizontal rule underneath

\chapter{Introduzione}
\label{Introduzione}
%
\section{Obiettivi}
La motivazione di questa tesi sta nella trovata collaborazione tra il \textit{Dipartimento di Ingegneria Industriale} dell'Università di Trento e \textit{AnteMotion S.r.l.}, azienda specializzata in realtà virtuale e simulazione di veicoli multibody. In particolare il modello di veicolo e pnumatico precedentemente studiati da \citeauthor{Larcher} in \cite{Larcher} sarà integrato nel simulatore di guida in tempo reale di AnteMotion. Pertanto, lo sviluppo del modello è stato finalizzata a minimizzare i tempi di compilazione massimizzando l'accuratezza del modello. La necessità di sviluppare un algoritmo che calcoli i parametri dell'interazione tra terreno e pneumatico getta le basi per il lavoro svolto.
%
\section{Il problema}
La \textit{simulazione} risolve alcuni dei problemi relativi al mondo della progettazione in modo sicuro ed efficiente, senza la necessità di costruire un prototipo dell'oggetto fisico. A differenza della modellazione fisica, che può coinvolgere il sistema reale o una copia in scala di esso, la simulazione è basata sulla tecnologia digitale e utilizza algoritmi ed equazioni per rappresentare il mondo reale al fine di imitare l'esperimento reale. Ciò comporta diversi vantaggi in termini di tempo, costi e sicurezza. Infatti, il modello digitale può essere facilmente riconfigurato e analizzato, mentre questo è solitamente impossibile o troppo oneroso del punto di vista di tempi e costi da fare con il sistema reale \cite{Anu}. Al giorno d'oggi esistono numerosi modelli di veicolo e pneumatico, certamente, più semplice è il modello, più veloce è la risoluzione delle equazioni costituenti e, a seconda delle applicazioni, deve essere scelta la giusta complessità per il modello. Per la maggior parte delle applicazioni di guida autonoma, un modello semplice è sufficiente per caratterizzare con un livello di dettaglio sufficiente il comportamento del veicolo, e poiché queste analisi sono molto spesso fatte con l'ausilio di \ac{HIL}, il modello dinamico del veicolo deve essere risolto in tempo reale con tipico passo di tempo di 1 millisecondo. Il vincolo in tempo reale implica un modello di veicolo di calcolo veloce, ciò significa che i modelli semplici con pochi parametri, di solito modelli lineari a singola traccia, sono particolarmente adatti per questo tipo di applicazioni. Tuttavia ci sono alcune situazioni che richiedono modelli più dettagliati, come ad esempio l'azione prodotta da un \ac{ADAS}, ovvero una manovra di sicurezza come l'elusione degli ostacoli o una frenata di emergenza, poiché il veicolo è spinto nella maggior parte dei casi al limite delle sue prestazioni \cite{impacts}. In queste condizioni di guida si devono tenere conto di molti fattori come ad esempio il comportamento degli pneumatici, che si sposta nella regione non lineare e i fenomeni transitori non sono più trascurabili. Ciò significa che un modello più dettagliato di quello utilizzato per la guida in condizioni "standard". L'accuratezza dinamica del modello è di grande importanza per ricavare previsioni realistiche delle prestazioni del veicolo e del sistema di controllo. È importante notare che modellare in modo esaustivo tutti i sistemi di un'auto sarebbe un compito estremamente arduo e a volte anche impossibile. Esistono quindi modelli empirici come il modello della \textit{MagicFormula} di Hans Pacejka e il modello \textit{Fiala} che cercano di imitare il reale comportamento del sistema. Il calcolo dei parametri di questo tipo di modelli richiede l'interpolazione di un set di dati di grandi dimensioni, e può quindi essere numericamente inefficiente o comunque troppo oneroso in termini di tempo.\\
Per studiare il comportamento del sistema in diversi scenari di guida, i moderni strumenti di simulazione spesso richiedono una grande quantità di di dati e l'unico modo per ottenerli è esecuzione stessa di migliaia di simulazioni. A questo proposito è quindi necessario un conducente virtuale o artificiale in grado di controllare il veicolo fino ai limiti di guidabilità. Pertanto, la caratteristica principale di un driver artificiale o virtuale è la capacità di guidare una varietà di veicoli con diverse caratteristiche dinamiche. Indipendentemente dall'architettura e dalla metodologia utilizzata per sviluppare il conducente artificiale, deve utilizzare una sorta di modello dinamico che rappresenta il comportamento del veicolo controllato. Si tratta dunque di una sorta di modello di comportamento dinamico del veicolo.\\
Lo scopo di questo lavoro si collega a quello già svolto da \citeauthor{Larcher} in \cite{Larcher}, dove grazie a un modello di veicolo completo con 14 gradi di libertà ha fornito un modello in grado di catturare con un livello di dettaglio appropriato il comportamento del veicolo quando viene spinto ai suoi limiti di maneggevolezza. La necessità di calcolare in tempo reale i parametri di input per il modello di ruota scelto da \cite{Larcher} definisce l'obiettivo di questo lavoro. Ovvero di avere una libreria scritta in \texttt{C++}, che con alcuni semplici parametri in input come la denominazione \ac{ETRTO} dello pneumatico e la posizione nello spazio, calcola i dati relativi all'interazione pneumatico strada quali l'intersezione del punto sotto il centro ruota, l'area di contatto, e l'inclinazione locale del piano strada. Il tutto cercando di minimizzare i tempi di compilazione.
%
\cite{Moller} \cite{RayTriangle} \cite{Swift} \cite{Schmeitz} 
\chapter{Descrizione della superficie stradale}
\label{rdf}
%
Oltre allo pneumatico, la superficie stradale rappresenta il secondo importante elemento che definisce il contatto. Perché una superficie stradale possa essere facilmente utilizzata da un in una simulazione deve essere prima discretizzata. La discretizzazione in questo caso avviene mediante la rappresentazione della superficie stessa in una triangolazione (\textit{mesh}). La \textit{mesh} è contenuta in un \textit{file} di formato \ac{RDF}, che contiene le posizioni $(x,y,z)$ di ogni vertice e i numeri di identificazione per ognuno dei tre vertici del triangolo, per ogni triangolo.\\
\indent
È importante notare che la discretizzazione del manto stradale è un processo molto importante in quanto, se campionato troppo grossolanamente potrebbe influire negativamente sui risultati dei calcoli per l'estrazione del piano strada locale. In altre parole, una semplificazione eccessiva, potrebbe causare degli errori tali da incorrere in risultati troppo approssimativi e non rispecchianti la realtà. Al contrario, una \textit{mesh} troppo fitta, aumenterebbe inutilmente i calcoli da eseguire, dilatando quindi i tempi di esecuzione. È bene quindi discretizzare più densamente in maniera oculata e solo dove occorre realmente, ovvero in prossimità di cordoli, marciapiedi o qualsiasi tipo di ostacolo che potrebbe influire sulle prestazioni della vettura.

\section{Il formato RDF per le superfici stradali}
%
\subsection{Superfici semplici}
Sfortunatamente, non esistono \textit{standard} universalmente riconosciuti per il formato RDF. In linea di massima le superfici stradali sono definite nei \textit{Road Data File} (\texttt{*.rdf}). Questa tipologia di \textit{file} è composta da varie sezioni, indicate da parentesi quadre.
\begin{pseudoc}
	{ Comments section }
	
	[UNITS]
	LENGTH = 'meter'
	ANGLE = 'degree'
	
	[MODEL]
	ROAD\_TYPE = '...'
	
	[PARAMETERS]
	...
\end{pseudoc}
Nella sezione \texttt{[UNITS]}, vengono impostate le unità di misura utilizzate nel \textit{file}. La sezione \texttt{[MODEL]} viene invece utilizzata per specificare la morfologia della superfice stradale, che può essere del tipo:
\begin{itemize}
	\item \texttt{ROAD\_TYPE = 'flat'}: superficie stradale piana.
	\item \texttt{ROAD\_TYPE = 'plank'}: singolo scalino o dosso orientato perpendicolarmente o obliquo rispetto all'asse $X$, con o senza bordi smussati.
	\item \texttt{ROAD\_TYPE = 'poly\_line'}: altezza della strada è in funzione della distanza percorsa.
	\item \texttt{ROAD\_TYPE = 'sine'}: superficie stradale costituita da una o più onde sinusoidali con lunghezza d'onda costante.
\end{itemize}
La sezione \texttt{[PARAMETERS]} contiene i parametri generali e specifici per il tipo di superficie stradale. Possono essere:
\begin{itemize}
	\item Generali:
	\begin{itemize}
		\item \texttt{MU}: è il fattore di correzione dell'attrito stradale (non il valore dell'attrito stesso), da moltiplicare con i fattori di ridimensionamento \texttt{LMU} del modello di pneumatico.\\
		Impostazione predefinita: \texttt{MU = 1.0}.
		\item \texttt{OFFSET}: è l'offset verticale del terreno rispetto al sistema di riferimento inerziale.
		\item \texttt{ROTATION\_ANGLE\_XY\_PLANE}: è l'angolo di rotazione del piano $XY$ attorno all'asse $Z$ della strada, ovvero la definizione dell'asse $X$ positivo della strada rispetto al sistema di riferimento inerziale.
	\end{itemize}
	\item Strada con scalino:
	\begin{itemize}
		\item \texttt{HEIGHT}: altezza dello scalino.
		\item \texttt{START}: distanza lungo l'asse $X$ della strada dell'inizio dello scalino.
		\item \texttt{LENGTH}: lunghezza dello scalino (escluso lo smusso) lungo l'asse $X$ della strada.
		\item \texttt{BEVEL\_EDGE\_LENGTH}: lunghezza del bordo smussato a $45°$ dello scalino.
		\item \texttt{DIRECTION}: rotazione dello scalino attorno all'asse $Z$, rispetto all'asse $Y$ della strada.\\
		Se lo scalino è posizionato trasversalmente, \texttt{DIRECTION = 0}. Se lo scalino è posto lungo l'asse $X$, \texttt{DIRECTION = 90}.
	\end{itemize}
	\item Polilinea:\\
	Il blocco \texttt{[PARAMETERS]} deve avere un sotto blocco chiamato \texttt{(XZ\_DATA)} e costituito da tre colonne di dati numerici:
	\begin{itemize}
		\item La colonna 1 è un insieme di valori $X$ in ordine crescente.
		\item Le colonne 2 e 3 sono insiemi di rispettivi valori $Z$ per la traccia sinistra e destra.
	\end{itemize}
	Esempio:
	\begin{pseudoc}
	[PARAMETERS]
	MU = 1.0
	OFFSET = 0.0
	ROTATION_ANGLE_XY_PLANE = 0.0 
	
	{ X_road	Z_left	Z_right }
	(XZ_DATA)
	-1.0e04	0	0
	0.0500	0	0
	0.1000	0	0
	0.1500	0	0
	... ... ...
	\end{pseudoc}
	\item Sinusoide:\\
	La strada a superficie sinusoidale è implementata come:
	\begin{equation}
	z(x)=\frac{H}{2}\left( 1 - \cos \left( \frac{2\pi \cdot (x-x_i)}{L} \right)   \right) 
	\end{equation}
	dove	
	\begin{itemize}
	 	\item $z$: coordinata verticale della strada;
	 	\item $H$: altezza;
	 	\item $x$: posizione attuale;
	 	\item $x_i$: inizio dell'onda sinusoidale;
	 	\item $L$: semi-periodo dell'onda sinusoidale.
	\end{itemize}
	I parametri sono:	
	\begin{itemize}
		\item \texttt{HEIGHT}: altezza dell'onda sinusoidale.
		\item \texttt{START}: distanza lungo l'asse $X$ della strada dall'inizio dell'onda sinusoidale.
		\item \texttt{LENGTH}: lunghezza dell'onda sinusoidale lungo l'asse $X$ della strada.
		\item \texttt{DIRECTION}: rotazione dell'onda sinusoidale attorno all'asse $Z$, rispetto all'asse $Y$ della strada.\\
		Se l'onda sinusoidale è posizionata trasversalmente, \texttt{DIRECTION = 0}. Se l'onda sinusoidale è posta lungo l'asse $X$, \texttt{DIRECTION = 90}.
	\end{itemize}
\end{itemize}
%
\subsection{Superfici complesse}
Sfortunatamente, queste informazioni appena descritte permettono di costruire strade troppo approssimate, che non rispecchiano la realtà. È quindi necessario inserire i risultati della discretizzazione della superficie stradale sopra citati.

Per descrivere la superficie stradale si utilizzerà dunque una \textit{mesh} poligonale. Quest'ultima può essere rappresentata utilizzando diversi metodi per memorizzare i dati dei vertici, bordi e facce. Nel caso specifico si andrà ad utilizzare una rappresentazione del tipo faccia-vertice. La \textit{mesh} faccia-vertice rappresenta un oggetto come un insieme di facce e un insieme di vertici. Questa rappresentazione è generalmente la più utilizzata in quanto permette una ricerca esplicita dei vertici di una faccia e delle facce che circondano un vertice.

Per descrivere una superficie stradale composta da una \textit{mesh} di triangoli si utilizzerà quindi la seguente struttura dati:
\begin{itemize}
	\item \texttt{[NODES]} (Vertici): presenti nella prima sezione, vengono descritti sotto forma di una quartina $(id,x,y,z)$ data dal numero di identificazione e dalle coordinate nello spazio.
	\item \texttt{[ELEMENTS]} (Facce): presenti nella seconda sezione, vengono descritti sotto forma di una quartina $(n_1,n_2,n_3,\mu)$ data dai numeri di identificazione dei tre vertici componenti $i$-esimo triangolo e dal coefficiente di attrito presente nella faccia.
\end{itemize}
Esempio:
\begin{pseudoc}
	[NODES]
	{ id x_coord y_coord z_coord }
	0 2.64637 35.8522 -1.59419e-005 
	1 4.54089 33.7705 -1.60766e-005 
	2 4.52126 35.8761 -1.62482e-005 
	3 2.66601 33.7456 -1.57714e-005 
	4 0.771484 35.8282 -1.56367e-005 
	5 0.791126 33.7206 -1.5465e-005
	... ... ... ...
	
	[ELEMENTS]
	{ n1 n2 n3 mu }
	1 2 3 1.0 
	2 1 4 1.0 
	5 4 1 1.0 
	... ... ... ...
\end{pseudoc}
Ulteriori parametri possono essere aggiunti prima della dichiarazione dei nodi della \textit{mesh}, come ad esempio:
\begin{itemize}
	\item \texttt{X\_SCALE}: riscala i punti delle coordinate dei nodi lungo l'asse $X$;
	\item \texttt{Y\_SCALE}: riscala i punti delle coordinate dei nodi lungo l'asse $Y$;
	\item \texttt{Z\_SCALE}: riscala i punti delle coordinate dei nodi lungo l'asse $Z$;
	\item \texttt{ORIGIN}: definisce la posizione dell'origine del sistema di riferimento della superficie stradale;
	\item \texttt{UP}: definisce la direzione positiva dell'asse $Z$;
	\item \texttt{[ORIENTATION]}: ruota i punti delle coordinate dei nodi secondo la matrice definita.
\end{itemize}
Esempio:
\begin{pseudoc}
	X_SCALE
	1000.0
	Y_SCALE
	1000.0
	Z_SCALE
	1000.0
	ORIGIN
	0 0 0
	UP
	0.0,0.0,1.0
	ORIENTATION
	1.0  0.0  0.0
	0.0  1.0  0.0
	0.0  0.0  1.0
\end{pseudoc}
%
\section{Analisi sintattico-grammaticale del formato RDF}
%
L'analisi sintattico-grammaticale è un processo che analizza un flusso continuo di dati in ingresso (letti per esempio da un \textit{file}) in modo da determinare la correttezza della sua struttura grazie ad una data grammatica formale. Il programma che esegue questo compito viene chiamato \textit{parser}. Nella maggior parte dei casi l'analisi sintattica opera su una sequenza di \textit{tokens} in cui l'analizzatore lessicale spezzetta l'\textit{input}.

Nel lavoro svolto è stato creato un algoritmo per eseguire l'analisi sintattico-grammaticale dei \textit{file} di tipo \ac{RDF}. Purtroppo, come precedentemente affermato, non esiste uno \textit{standard} universalmente riconosciuto per questo formato. Creare dunque un \textit{parser} o definire un generatore di \textit{parser} è arduo. Si è quindi optato per la creazione di un programma che rilevi solo i nodi (\texttt{[NODES]}), li salvi temporaneamente e, dopo aver immagazzinato anche i dati relativi agli elementi (\texttt{[ELEMENTS]}), istanzi un oggetto di tipo \textit{mesh}, composto dai nodi dichiarati nella sezione degli elementi. Gli altri parametri non sono stati considerati.

\begin{figure}
	\centering
	\includegraphics[width=0.7\linewidth]{Figures/mesh}
	\caption{Esempio di superficie rappresentata tramite \textit{mesh} triangolare.}
\end{figure}
\begin{figure}
	\centering
	\includegraphics[width=0.7\linewidth]{Figures/mesh_1}
	\caption{Intersezione stradale rappresentata tramite \textit{mesh} triangolare.}
\end{figure}

Come verrà richiamato nelle conclusioni, l'importanza di definire uno \textit{standard} per il formato \ac{RDF} è di cruciale importanza. In questo modo si potrà creare un generatore di \textit{parser} con una grammatica e un lessico ben definiti.
%\chapter{Algoritmi}
\label{Geom_Algos}
%
\section{Parsificazione}
%
\subsection{Introduzione}
La parsificazione o analisi sintattica è un processo che analizza un flusso continuo di dati in ingresso (letti per esempio da un file o una tastiera) in modo da determinare la correttezza della sua struttura grazie ad una data grammatica formale. Un \textit{parser} è un programma che esegue questo compito. Nella maggior parte dei casi, l'analisi sintattica opera su una sequenza di \textit{token} in cui l'analizzatore lessicale spezzetta l'input.
%
\subsection{Parsificazione del formato RDF}
Nel lavoro svolto è stato creato un algoritmo per pardificare i file di tipo \ac{RDF} che descrivono superfici complesse. Purtroppo, come precedentemente detto, non esiste uno standard universalmente riconosciuto per questo formato. Creare dunque un \textit{parser} o definire un generatore di parser è arduo. Si è quindi optato per la creazione di un \textit{parser} che rilevi solo i nodi (\texttt{[NODES]}), li salvi temporaneamente e, dopo aver immagazzinato anche i dati relativi agli elementi (\texttt{[ELEMENTS]}), instanzi un oggetto \textit{mesh}, composto dai nodi dichiarati nella sezione elementi. Gli altri parametri non sono stati considerati.\\

Come verrà richiamato nelle conclusioni, l'importanza di definire uno standard per il formato \ac{RDF} è di cruciale importanza. In questo modo si potrà creare un generatore di parser con una grammatica e un lessico ben definiti, nonché aumentarne l'efficienza e la stabilità.
%
%
\section{\textit{Bounding Volume Hierarchy}}
%
\subsection{Introduzione}
Una \ac{BVH} è una struttura ad albero su un insieme di oggetti geometrici. Tutti gli oggetti geometrici sono raccolti in volumi limite che formano i nodi fogliari dell'albero. Questi nodi vengono quindi raggruppati come piccoli insiemi e racchiusi in volumi di delimitazione più grandi. Questi, a loro volta, sono ancora raggruppati e racchiusi in altri volumi di delimitazione più grandi in modo ricorsivo, risultando infine in una struttura ad albero con un singolo volume di delimitazione nella parte superiore dell'albero. Le gerarchie di volumi limitanti vengono utilizzate per supportare in modo efficiente diverse operazioni su insiemi di oggetti geometrici, come ad esempio il rilevamento delle collisioni.

Sebbene il \textit{wrapping} degli oggetti nei volumi di delimitazione e l'esecuzione di test di collisione su di essi prima del test della geometria dell'oggetto stesso semplifichino i test e possano comportare miglioramenti significativi delle prestazioni, è ancora in corso lo stesso numero di test a coppie tra volumi di delimitazione. Organizzando i volumi di delimitazione in una gerarchia di volumi di delimitazione, la complessità temporale (il numero di test eseguiti) può essere ridotta logaritmicamente nel numero di oggetti. Con una tale gerarchia in atto, durante i test di collisione, i volumi secondari non devono essere esaminati se i loro volumi principali non sono intersecati.
%
\subsection{\textit{Minimum Bounding Box}}
In geometria, il rettangolo minimo o più piccolo (o \ac{MBB}) per racchiudere un insieme di punti $S$ in $N$ dimensioni è l'rettangolo con la misura più piccola (area, volume o ipervolume in dimensioni superiori) all'interno del quale si trovano tutti i punti.  Il termine "iper-rettangolo (o più semplicemente \textit{box}) deriva dal suo utilizzo nel sistema di coordinate cartesiane, dove viene effettivamente visualizzato come un rettangolo (caso bidimensionale), parallelepipedo rettangolare (caso tridimensionale), ecc. Nel caso bidimensionale viene chiamato rettangolo di delimitazione minimo.
%
\subsubsection{\textit{Axis Aligned Bounding Box}}
Il \ac{MBB} allineato agli'assi (\ac{AABB}) per un determinato set di punti è il rettangolo di delimitazione minimo soggetto al vincolo che i bordi del rettangolo sono paralleli agli assi cartesiani. È il prodotto cartesiano di $N$ intervalli ciascuno dei quali è definito da un valore minimo e un valore massimo della coordinata corrispondente per i punti in $S$.

I rettangoli di delimitazione minimi allineati all'asse vengono utilizzati per determinare la posizione approssimativa di un oggetto e come descrittore molto semplice della sua forma. Ad esempio, nella geometria computazionale e nelle sue applicazioni quando è necessario trovare intersezioni nel set di oggetti, il controllo iniziale sono le intersezioni tra i loro \ac{MBB}. Dato che di solito è un'operazione molto meno costosa del controllo dell'intersezione effettiva (perché richiede solo confronti di coordinate), consente di escludere rapidamente i controlli delle coppie che sono molto distanti.

\begin{figure}[h]
	\centering
	\includegraphics[width=\linewidth]{Figures/AABB}
	\caption{Esempio di albero di tipo \ac{AABB}.}
	\label{AABB}
\end{figure}
%
\subsubsection{\textit{Arbitrarily Oriented Bounding Box}}
Il \ac{MBB} orientato arbitrariamente (\ac{AOBB}) è il rettangolo di delimitazione minimo, calcolato senza vincoli per quanto riguarda l'orientamento del risultato. Gli algoritmi del rettangolo di delimitazione minimo basati sul metodo dei calibri rotanti possono essere utilizzati per trovare l'area di delimitazione dell'area minima o del perimetro minimo di un poligono convesso bidimensionale in tempo lineare e di un punto bidimensionale impostato nel tempo impiegato costruire il suo scafo convesso seguito da un calcolo del tempo lineare. Un algoritmo di pinze rotanti tridimensionali può trovare il rettangolo di delimitazione orientato arbitrariamente sul volume minimo di un punto tridimensionale impostato in tempo cubo.
%
\subsubsection{\textit{Object Oriented Bounding Box}}
Nel caso in cui un oggetto abbia un proprio sistema di coordinate locale, può essere utile memorizzare un rettangolo di selezione relativo a questi assi, che non richiede alcuna trasformazione quando cambia l'orientazione dell'oggetto stesso.

%\begin{figure}[h]
%	\centering
%	\includegraphics[width=0.9\linewidth]{Figures/BB}
%	\caption{Tipi di \ac{BVH}.}
%	\label{BB}
%\end{figure}
%
\subsection{Intersezione tra Alberi AABB}

Per il rilevamento delle collisioni tra oggetti in due dimensioni, l'intersezione tra alberi di tipo \ac{AABB}, è l'algoritmo più veloce per determinare se le due entità di gioco si sovrappongono o meno, e in che parti. Nello specifico, ciò consiste nel controllare le posizioni delle \textit{i}-esime \ac{BB} nello spazio delle coordinate bidimensionali per vedere se si sovrappongono.

Il vincolo di allineamento dei rettangoli agli assi è presente per motivi di prestazioni, infatti, l'area di sovrapposizione tra due riquadri non ruotati può essere controllata solo con confronti logici. Mentre i riquadri ruotati richiedono ulteriori operazioni trigonometriche, che sono più lente da calcolare. Inoltre, se si hanno entità che possono ruotare, le dimensioni dei rettangoli e/o sotto-rettangoli dovranno modificarsi in modo da avvolgere ancora l'oggetto o si dovrà optare per un altro tipo di geometria di delimitazione, come le sfere (che sono invarianti alla rotazione).

Nel caso specifico, l'ombra dello pneumatico sarà rappresentata da un albero di tipo \ac{AABB} con una sola foglia. Ovvero si andrà a rappresentare lo pneumatico con una \ac{BB} avente lati uguali e rappresentanti il massimo ingombro che può avere nello spazio. Si andrà inoltre ad incrementare del 10\% ognuno di questi lati in modo da tenere conto dell'angolo di camber, che portrebbe portare i punti di campionamento del terreno fuori dall'ombra. La strada, contrariamente al pneumatico, verrà tenuta come riferimento assoluto. In altre parole, una volta effettuato la parsificazione del file \ac{RDF}, verrà calcolato l'albero di tipo \ac{AABB}. Lo pneumatico si muoverà all'interno della \textit{mesh} e la sua ombra verrà ricalcolata e intersecata con l'albero \ac{AABB} per ottenere tutti i triangoli in corrispondenza della stessa.

Volendo intersecare due semplici \ac{BB}, quali $A = \left[ \texttt{A.minX}, \texttt{A.maxX} ;  \texttt{A.minY}, \texttt{A.maxY} \right]$ e $B = \left[ \texttt{B.minX}, \texttt{B.maxX} ;  \texttt{B.minY}, \texttt{B.maxY} \right]$, verrà usata la seguente funzione.
\begin{pseudoc}
	function intersect(A, B) {
		return (A.minX <= B.maxX && A.maxX >= B.minX) &&
					 (A.minY <= B.maxY && A.maxY >= B.minY);
	}
\end{pseudoc}
\noindent
Volendo intersecare un albero di tipo \ac{AABB} e una semplice \ac{BB}, basterà ripetere a più step la funzione precedente lungo i rami dell'albero. Una volta arrivati a una o più foglia avremo tutti gli oggetti (o triangoli nel caso specifico) che sono posti in corrispondenza della \ac{BB} (od ombra dello pneumatico nel caso specifico). Questi triangoli verranno poi usati per determinare il piano strada locale e il punto di contatto virtuale dello pneumatico.

È imporatante notare che il metodo appena visto, presenta numerosi vantaggi.
\begin{itemize}
	\item Riduzione del numero di comparazioni da effettuare per ottenere l'intersezione \ac{BB}-albero \ac{AABB}. Infatti, la \textit{mesh} può contenere decine di migliaia di trangoli, il metodo presentato consente di ridurre logarirmicamente il numero di comparazioni necessarie per ottenere il risultato.
	\item Riduzione del numero di trangoli da processare per ottenere il piano strada locale e il punto di contatto virtuale dello pneumatico. Infatti, vengono solamente processati quelli posti in corrispondenza del'ombra dello pneumatico.
\end{itemize}
%
\section{Algoritmi Geometrici}
%
\subsection{Introduzione}
La geometria computazionale è la branca dell'informatica che studia le strutture dati e gli algoritmi efficienti per la soluzione di problemi di natura geometrica e la loro implementazione al calcolatore. Storicamente, è considerato uno dei campi più antichi del calcolo, anche se la geometria computazionale moderna è uno sviluppo recente. La ragione principale per lo sviluppo della geometria computazionale è stata dovuta ai progressi compiuti nella computer grafica, \ac{CAD}, \ac{CAM} e nella visualizzazione matematica. Ad oggi, le applicazioni della geometria computazionale si trovano nella robotica, nella progettazione di circuiti integrati, nella visione artificiale, in \ac{CAE} e nel \ac{GIS}. I rami principali della geometria computazionale sono:
\begin{itemize}
	\item \textit{Calcolo combinatorio} (o \textit{geometria algoritmica}), che si occupa di oggetti geometrici come entità discrete. Ad esempio, può essere utilizzato per determinare il poliedro o il poligono più piccolo che contiene tutti i punti forniti, o più formalmente, dato un insieme di punti, si deve determinare il più piccolo insieme convesso che li contenga tutti (problema dell'inviluppo convesso).
	\item \textit{Geometria di calcolo} numerica (o \ac{CAGD}), che si occupa principalmente di rappresentare oggetti del mondo reale in forme adatte per i calcoli informatici nei sistemi \ac{CAD} e \ac{CAM}. Questo ramo può essere visto come uno sviluppo della geometria descrittiva ed è spesso considerato un ramo della computer grafica o del \ac{CAD}. Entità importanti di questo ramo sono superfici e curve parametriche, come ad esempio le \textit{spline} e \textit{curve di Bézier}.
\end{itemize}

In questo capitolo tutti gli algoritmi che verranno utilizzati in seguito durante l'analisi geometrica dell'intersezione tra pneumatico e superficie stradale saranno trattati. Questi algoritmi sono la soluzione di alcuni semplici ma molto importanti problemi, che devono essere risolti in modo efficiente. In particolare le intersezioni tra:
\begin{itemize}
	\item punto e segmento (sul piano);
	\item punto e circonferenza (sul piano);
	\item raggio e circonferenza (sul piano);
	\item raggio e triangolo (sullo spazio);
\end{itemize}
saranno esaminati al fine di trovare la massima prestazione in termini di efficienza computazionale.
%
\subsection{Intersezione tra Entità Geometriche}
%
\subsubsection{Punto-Segmento}
Dato un punto $P = (x_p, y_p)$ e un segmento definito da due punti $A = (x_A, y_B)$ e $B = (x_B, y_B)$.

\begin{figure}[h!]
	\centering
	\begin{tikzpicture}
	\def\r{2};
	\coordinate (P) at (0.3,0.7);
	\coordinate (A) at (-2.0,0.0);
	\coordinate (B) at (+2.0,0.0);
	\draw[fill] (A) circle [radius=1pt] node[above] {$A$};
	\draw[fill] (B) circle [radius=1pt] node[above] {$B$};
	\draw[fill] (P) circle [radius=1pt] node[above] {$P$};
	\draw[thick](A) -- (B);
	\end{tikzpicture}
	\caption{Schema grafico per l'intersezione punto-segmento}
\end{figure}
\noindent
Per determinare se il punto $P$ è intermo al segmento si eseguiranno i seguenti step.
\begin{enumerate}
	\item Creazione di un vettore $\vv{AB}$ e di un vettore $\vv{AP}$.
	\item Calcolo il prodotto vettoriale  $\vv{P_1P_2} \times  \vv{PP_1}$, se il modulo del vettore risultante è nullo allora il punto $P$ appartiene al segmento considerato.
	\item Calcolo il prodotto scalare tra $\vv{AB}$ e $\vv{AP}$. Se è nullo allora il punto $P$ è coincidente a $A$, se è pari al modulo di $\vv{AB}$ allora il punto $P$ è coincidente a $B$, se è compreso tra 0 il modulo di $\vv{AB}$, allora il punto $P$ giace all'interno del segmento considerato.
\end{enumerate}
Il codice che esegue questo tipo di test è riportato in \figurename{ \ref{pointsegment}}

\begin{figure}[h!]
	\hfill
	\begin{subfigure}{.45\textwidth}
		\centering
		\begin{tikzpicture}
		\coordinate (P0) at (0.3,0.7);
		\coordinate (P3) at (-0.7,0);
		\coordinate (A) at (-2.0,0.0);
		\coordinate (B) at (+2.0,0.0);
		\draw[fill] (P0) circle [radius=1pt] node[above] {\texttt{0}};
		\draw[fill] (A) circle [radius=1pt] node[above] {\texttt{1}};
		\draw[fill] (B) circle [radius=1pt] node[above] {\texttt{2}};
		\draw[fill] (P3) circle [radius=1pt] node[above] {\texttt{3}};
		\draw[thick](A) -- (B);
		\end{tikzpicture}
		\caption{Output tipo \texttt{integer}}
	\end{subfigure}
	\hfill
	\begin{subfigure}{.45\textwidth}
		\centering
		\begin{tikzpicture}
		\coordinate (P0) at (0.3,0.7);
		\coordinate (P3) at (-0.7,0);
		\coordinate (A) at (-2.0,0.0);
		\coordinate (B) at (+2.0,0.0);
		\draw[fill] (P0) circle [radius=1pt] node[above] {\texttt{false}};
		\draw[fill] (A) circle [radius=1pt] node[above] {\texttt{true}};
		\draw[fill] (B) circle [radius=1pt] node[above] {\texttt{true}};
		\draw[fill] (P3) circle [radius=1pt] node[above] {\texttt{true}};
		\draw[thick](A) -- (B);
		\end{tikzpicture}
		\caption{Output tipo \texttt{bool}}
	\end{subfigure}
	\hfill
	\caption{Schemi per l'output dell'intersezione punto-segmento.}
\end{figure}

\begin{figure}[h!]
	\hfill
	\begin{subfigure}[t]{.45\linewidth}
	\raggedright
	Output tipo \texttt{integer}\\
	\vspace{.5em}
	\begin{pseudoc}
	if (AB.cross(AP) > epsilon) { return 0 }
	KAP = AB.dot(AP)
	if ( KAP      < -epsilon ) { return 0 }
	if ( abs(KAP) < epsilon  ) { return 1 }
	KAB = AB.dot(AB)
	if ( KAP > KAB )  { return 0 }
	if ( abs(KAP - KAB) < epsilon ) { return 2 }
	return 3 // The point is on the segment
	\end{pseudoc}
	\end{subfigure}
	\hfill
	\begin{subfigure}[t]{.45\textwidth}
	\raggedright
	Output tipo \texttt{bool}\\
	\begin{pseudoc}
	if (AB.cross(AP) > epsilon) { return false }
	KAP = AB.dot(AP)
	if ( KAP      < -epsilon ) { return false }
	if ( abs(KAP) < epsilon  ) { return true }
	KAB = AB.dot(AB)
	if ( KAP > KAB )  { return false }
	if ( abs(KAP - KAB) < epsilon ) { return true }
	return true // The point is on the segment
	\end{pseudoc}
	\end{subfigure}
	\hfill
	\caption{Schema del codice per l'intersezione punto-segmento.}
	\label{pointsegment}
\end{figure}
%
\subsubsection{Punto-Cerchio}
Having a circle with center $C = (x_c, y_c)$ and radius $r$, the problem consists in finding out whether a query point $P = (x_p, y_p)$ is inside, outside or on the circle.
%
\begin{figure}
	\centering
	\begin{tikzpicture}
	\def\r{2};
	\coordinate (C) at (0,0) node[above left] {$C$};
	\draw[thick, fill=gray!10](C) circle (\r);
	\coordinate (P) at (-0.5,-1.5);
	\draw[fill] (C) circle [radius=1pt];
	\draw[fill] (P) circle [radius=1pt];
	\draw(C) -- (P)  node[above left] {$P$} node[pos=0.5, right] {$d$};
	\draw(C) -- ({sqrt(\r)},{sqrt(\r)}) node[pos=0.4, above] {$r$};
	\end{tikzpicture}
	\caption{Point-circle intersection problem scheme.}
\end{figure}
%
The solution to the problem is simple: the distance between the circle center $C$ and the query point $P$ is given by the \textit{Pythagorean theorem} as
\begin{equation}
	d=\sqrt{(x_p-x_c)^2 + (y_p-y_c)^2}
\end{equation}
The query point $P$ is \textit{inside} the circle if $d<r$, on the circle if $d = r$, and \textit{outside} the circle if $d > r$. Little work can be saved by comparing $d^2$ with $r^2$ instead: the point $P$ is \textit{inside} the circle if $d^2<r^2$, on the circle if $d^2 = r^2$, and \textit{outside} the circle if $d^2 > r^2$. Thus, the final comparison will be between the number $(x_p-x_c)^2 + (y_p-y_c)^2$ and $r^2$.\\
The \textit{inputs} of the point-circle intersection algorithm are:
\begin{itemize}
	\item the circle center $C = (x_c, y_c)$;
	\item the circle radius $r$;
	\item a query point $P=(x_p, y_p)$.
\end{itemize}
The \textit{output} could be an integer which value is:
\begin{itemize}
	\item 0 if the point is outside;
	\item 1 if the point is inside;
	\item 2 if the point is on the circle.
\end{itemize}
Another option could be a boolean which value is:
\begin{itemize}
	\item false if the point is outside;
	\item true if the point is inside or on the circle.
\end{itemize}
%
\begin{figure}
\hfill
	\begin{subfigure}{.45\textwidth}
	\centering
	\begin{tikzpicture}
	\def\r{2};
	\coordinate (C) at (0,0);
	\draw[thick, fill=gray!10](C) circle (\r);
	\draw[fill] (0,-0.5) circle [radius=1pt] node[above left] {\texttt{1}};
	\draw[fill] (-\r,0) circle [radius=1pt] node[above left] {\texttt{2}};
	\draw[fill] (3,0) circle [radius=1pt] node[above left] {\texttt{0}};
	\end{tikzpicture}
	\caption{Output tipo \texttt{integer}}
	\end{subfigure}
\hfill
\begin{subfigure}{.45\textwidth}
	\centering
	\begin{tikzpicture}
	\def\r{2};
	\coordinate (C) at (0,0);
	\draw[thick, fill=gray!10](C) circle (\r);
	\draw[fill] (0,-0.5) circle [radius=1pt] node[above left] {\texttt{true}};
	\draw[fill] (-\r,0) circle [radius=1pt] node[above left] {\texttt{true}};
	\draw[fill] (3,0) circle [radius=1pt] node[above] {\texttt{false}};
	\end{tikzpicture}
	\caption{Output tipo \texttt{bool}}
\end{subfigure}
\hfill
\caption{Schemi per l'output dell'intersezione punto-cerchio.}
\end{figure}
On \figurename{ \ref{Pointcircle}} the schemes for the point-circle intersection algorithm with integer and boolean outputs are reported.\\
%
\begin{figure}[htbp]
\hfill
	\begin{subfigure}[t]{.45\linewidth}
	\raggedright
	Output tipo \texttt{integer}\\
	\vspace{.5em}
	\begin{pseudoc}
	d = (x_p-x_c)^2 + (y_p-y_c)^2
	if (d > r^2) { return  0}
	else if (d < r^2) { return 1 }
	else { return 2} // d = r^2
	\end{pseudoc}
	\end{subfigure}
\hfill
	\begin{subfigure}[t]{.45\textwidth}
	\raggedright
	Output tipo \texttt{bool}\\
	\vspace{.5em}
	\begin{pseudoc}
	d = (x_p-x_c)^2 + (y_p-y_c)^2
	if (d > r^2) { return  true}
	else { return false} // d <= r^2
	\end{pseudoc}	
	\end{subfigure}
\hfill
\caption{Schemi per l'intersezione punto-cerchio.}
\label{Pointcircle}
\end{figure}
%
\subsubsection{Piano-Cerchio}
%
\subsubsection{Piano-Triangolo}

\subsubsection{Raggio-Triangolo}
Having a triangle with vertices $(V_1,V_2,V_3)$ and a ray $R$ with origin $R_O$ and direction $R_D$, the problem consists in finding out whether the ray hits or not the triangle and if so, where is the intersection point $P$.
%
\begin{figure}[htbp]
	\centering
	\begin{tikzpicture}
	\coordinate [label=left:$V_1$] (A) at (-1,-0.5);
	\coordinate [label=above:$V_2$] (B) at (1,1);
	\coordinate [label=right:$V_3$] (C) at (2.2,-1);
	\draw[fill=gray!10] (A) -- (B) -- (C) -- (A);
	\def\xp{0.4};
	\def\yp{-0.2};
	\def\xd{-1};
	\def\yd{1};
	\def\mag{1.7};
	\coordinate [label=above right:$P$] (P) at (\xp,\yp);
	\coordinate [label={[shift={(0.1,0.1)}]$R_D$}] (RD) at (\xp+\xd,\yp+\yd);
	\coordinate [label=above:$R_O$] (RO) at (\xp+\xd*\mag,\yp+\yd*\mag);
	\draw[fill] (P) circle [radius=1pt];
	\draw[fill] (RO) circle [radius=1pt];
	\draw[-stealth] (RO) -- (RD);
	\draw[dashed] (RD) -- (P);
	\end{tikzpicture}
	\caption{Ray-triangle intersection problem scheme.}
\end{figure}
%
Over the last decades, plenty of algorithms for solving this problem had been purposed, so there are several solutions to the ray/triangle or ray-triangle intersection problem. Three of the most relevant algorithms are:
\begin{itemize}
	\item \textit{Badouel} algorithm;
	\item \textit{Segura} algorithm;
	\item \textit{M\"oller-Trumbore} algorithm.
\end{itemize}
As \citeauthor{RayTriangle} states in \cite{RayTriangle}, the M\"oller-Trumbore's is the faster algorithm when the normal and/or the projection plane have not been previously stored, as in this thesis.\\
%
\begin{figure}[htbp]
	\centering
	\hfill
	\begin{subfigure}[t]{.3\linewidth}
		\begin{tikzpicture}
			\coordinate (O) at (0,0,0);
			\def\axisl{3}
			\draw[-stealth] (O) -- (\axisl,0,0) node[below]{$x$};
			\draw[-stealth] (O) -- ({sqrt(\axisl)},{sqrt(\axisl)},0) node[above]{$y$};
			\draw[-stealth] (O) -- (0,\axisl,0) node[left]{$z$};
			
			\coordinate [label=below:$V_1$] (V1) at ({\axisl/1.5},0,0);
\coordinate [label=above:$V_2$] (V2) at ({sqrt(\axisl)/1.5},{sqrt(\axisl)/1.5},0);
\coordinate [label=above:$V_3$] (V3) at (0,0,0);
			
			\draw[fill=gray!10] (V1) -- (V2) -- (V3) -- (V1);

		\end{tikzpicture}
	\end{subfigure}
\hfill
\begin{subfigure}[t]{.3\linewidth}
	\begin{tikzpicture}
			\coordinate (O) at (0,0,0);
			\def\axisl{3}
		\draw[-stealth] (O) -- (\axisl,0,0) node[below]{$x$};
\draw[-stealth] (O) -- ({sqrt(\axisl)},{sqrt(\axisl)},0) node[above]{$z$};
\draw[-stealth] (O) -- (0,\axisl,0) node[left]{$y$};
			
			\coordinate [label=below:$V_1$] (V1) at ({\axisl/1.5},0,0);
			\coordinate [label=above:$V_2$] (V2) at ({sqrt(\axisl)/1.5},{sqrt(\axisl)/1.5},0);
			\coordinate [label=above:$V_3$] (V3) at (0,0,0);
			

			
			\draw[fill=gray!10] (V1) -- (V2) -- (V3) -- (V1);

	\end{tikzpicture}
\end{subfigure}
\hfill
\begin{subfigure}[t]{.3\linewidth}
	\begin{tikzpicture}
		\coordinate (O) at (0,0,0);
		\def\axisl{3}
		\draw[-stealth] (O) -- (\axisl,0,0) node[below]{$\alpha$};
		\draw[-stealth] (O) -- ({sqrt(\axisl)},{sqrt(\axisl)},0) node[above]{$\beta$};
		\draw[-stealth] (O) -- (0,\axisl,0) node[left]{$\gamma$};
		
		\coordinate [label={[below,font=\tiny]:$1$}] (V1) at ({\axisl/1.5},0,0);
		\coordinate [label={[shift={(-0.1,0.0)},font=\tiny]:$1$}] (V2) at ({sqrt(\axisl)/1.5},{sqrt(\axisl)/1.5},0);
		\coordinate (V3) at (0,0,0);
		
		\draw[fill=gray!10] (V1) -- (V2) -- (V3) -- (V1);
	\end{tikzpicture}
\end{subfigure}
\hfill
\caption{Transformation and base change of ray in M\"oller-Trumbore algorithm.}
\end{figure}

The inputs of the M\"oller-Trumbore algorithm are:
\begin{itemize}
	\item Triangle vertices $(V_1, V_2, V_3)$;
	\item Segment points $(Q_1, Q_2)$.
\end{itemize}
%
\begin{figure}[htbp]
\hfill
	\begin{subfigure}[t]{.4\linewidth}
		\raggedright
		\textit{With back-face culling}\\
		\vspace{.5em}
		$Q = Q_2 - Q_1$\\
		$E_1 = V_2 - V_1$\\
		$E_2 = V_3 - V_1$\\
		$A = Q \times E_2$\\
		$D = A \cdot E_1$\\
		$\mathbf{if} \, (D > \varepsilon) \{$\\
		\quad $T = Q_1 - V_1$\\
		\quad $u = A \cdot T$\\
		\quad $\mathbf{if} \, (u < 0.0 \, || \, u > D) \{$\\
		\quad \quad $\mathbf{return \, false}$\\
		\quad $\}$\\
		\quad $B = T \times E_1$\\
		\quad $v = B \cdot Q$\\
		\quad $\mathbf{if} \, (v < 0.0 \, || \, u + v > D) \{$\\
		\quad \quad $\mathbf{return \, false}$\\
		\quad $\}$\\
		$\} \, \mathbf{else \, if} \, (D < -\varepsilon) \{$\\
		\quad $T = Q_1 - V_1$\\
		\quad $u = A \cdot T$\\
		\quad $\mathbf{if} \, (u > 0.0 \, || \, u < D) \{$\\
		\quad \quad $\mathbf{return \, false}$\\
		\quad $\}$\\
		\quad $B = T \times E_1$\\
		\quad $v = B \cdot Q$\\
		\quad $\mathbf{if} \, (v > 0.0 \, || \, u + v < D) \{$\\
		\quad \quad $\mathbf{return \, false}$\\
		\quad $\}$\\
		$\} \, \mathbf{else} \, \{$\\
		\quad $\mathbf{return \, false}$\\
		$\}$\\
		$D_{inv} = 1.0 / D$\\
		$t = (B \cdot E_2) * D_{inv}$\\
		$\mathbf{if} \, (t > 0.0) \{$\\
		\quad $P = Q + D * t$\\
		\quad $\mathbf{return \, true}$\\
		$\} \, \mathbf{else} \, \{$\\
		\quad $\mathbf{return \, false}$\\
		$\}$\\
	\end{subfigure}
\hfill
	\begin{subfigure}[t]{.4\textwidth}
		\raggedright
		\textit{Without back-face culling}\\
		\vspace{.5em}
		$Q = Q_2 - Q_1$\\
		$E_1 = V_2 - V_1$\\
		$E_2 = V_3 - V_1$\\
		$A = Q \times E_2$\\
		$D = A \cdot E_1$\\
		$\mathbf{if} \, (D < \varepsilon) \{$\\
		\quad $\mathbf{return \, false}$\\
		$\}$\\
		$T = Q_1 - V_1$\\
		$u = A \cdot T$\\
		$\mathbf{if} \, (u < 0.0 \, || \, u > D) \{$\\
		\quad $\mathbf{return \, false}$\\
		$\}$\\
		$B = T \times E_1$\\
		$v = B \cdot Q$\\
		$\mathbf{if} \, (v < 0.0 \, || \, u + v > D) \{$\\
		\quad \quad $\mathbf{return \, false}$\\
		$\}$\\
		$D_{inv} = 1.0 / D$\\
		$t = (B \cdot E_2) * D_{inv}$\\
		$\mathbf{if} \, (t > 0.0) \{$\\
		\quad $P = Q + D * t$\\
		\quad $\mathbf{return \, true}$\\
		$\} \, \mathbf{else} \, \{$\\
		\quad $\mathbf{return \, false}$\\
		$\}$\\
	\end{subfigure}
\hfill
\caption{Ray-triangle intersection algorithm schemes.}
\end{figure}
\include{Chapters/chapterExamples}

\appendix
\chapter{Convenzioni e Notazioni}
\label{Notazioni}
%
\subsection{Sistemi di Riferimento}
La convenzione utilizzata per definire gli assi del sistema di riferimento della vettura è la \ac{ISO} 8855.

\begin{figure}[h!]
	\centering
	\includegraphics[width=0.8\linewidth]{Figures/iso_convention}
	\caption{Rappresentazione degli assi del sistema di riferimento della vettura secondo la convenzione ISO-V.}
	\label{isoconventionv}
\end{figure}

\noindent
Il sistema di riferimento della ruota è conforme alla convenzione \ac{ISO}-V, la cui disposizione degli assi è illustrata nella \figurename{ \ref{isoconventionc}}. L'origine del sistema di riferimento del vettore ruota è posta in corrispondenza del centro della ruota mentre posizione e orientamento relativi rispetto al sistema di riferimento del telaio sono definiti attraverso il modello della sospensione descritto in \cite{Larcher}.

\begin{figure}[h!]
	\centering
	\includegraphics[width=0.6\linewidth]{Figures/iso_convention_wheel}
	\caption{Rappresentazione degli assi  del sistema di riferimento dello pneumatico secondo la convenzione ISO-C.}
	\label{isoconventionc}
\end{figure}
%
\subsection{Matrice di Trasformazione}
Per descrivere sia l'orientamento che la posizione di un sistema di assi nello spazio, viene introdotta la matrice roto-traslazione, chiamata anche matrice di trasformazione. Questa notazione permette di impiegare le operazioni matrice-vettore per l'analisi di posizione, velocità e accelerazione. La forma generale di una matrice di trasformazione è del tipo:
%
\begin{equation}
T_m = \left[
\begin{array}{ccc|c}
& & & O_{mx}\\
\multicolumn{3}{c|}{\multirow{3}{*}{\raisebox{20mm}{\scalebox{1.5}{$[R_m]$}}}} & O_{my}\\
& & & O_{mz}\\ \hline
0 & 0 & 0 & 1
\end{array}\right]
\end{equation}\\
%
dove $R_m$ è la matrice di rotazione $3 \times 3$ del sistema di riferimento in movimento e $O_{mx}$, $O_{my}$ e $O_{mz}$ sono le coordinate della sua origine nel sistema di riferimento assoluto o nativo.

L'introduzione dell'elemento fittizio 1 nel vettore della posizione di origine e la successiva spaziatura interna zero della matrice rende possibili le moltiplicazioni matrice-vettore, rendendo la matrice di trasformazione una notazione compatta e conveniente per la descrizione dei sistemi di riferimento. Si noti che per i vettori, le informazioni traslazionali vengono trascurate imponendo l'elemento fittizio pari a 0.
%
\chapter{Codice della Libreria C++}
\label{LibraryCode}
%
\section{TireGround.hh}
\footnotesize
\renewcommand{\baselinestretch}{1.0}
\lstinputlisting[language=C++]{../TireGround/include/TireGround.hh}
\renewcommand{\baselinestretch}{1.25}
%
\section{RoadRDF.hh}
\renewcommand{\baselinestretch}{1.0}
\lstinputlisting[language=C++]{../TireGround/include/RoadRDF.hh}
\renewcommand{\baselinestretch}{1.25}
%
\section{RoadRDF.cc}
\renewcommand{\baselinestretch}{1.0}
\lstinputlisting[language=C++]{../TireGround/src/RoadRDF.cc}
\renewcommand{\baselinestretch}{1.25}
%
\section{PatchTire.hh}
\renewcommand{\baselinestretch}{1.0}
\lstinputlisting[language=C++]{../TireGround/include/PatchTire.hh}
\renewcommand{\baselinestretch}{1.25}
%
\section{PatchTire.cc}
\renewcommand{\baselinestretch}{1.0}
\lstinputlisting[language=C++]{../TireGround/src/PatchTire.cc}
\renewcommand{\baselinestretch}{1.25}
%
\chapter{Codice dei Tests}
\label{TestsCode}
%
\section{Tests Geometrici}
%
\subsection{Geometry-test1.cc}
\renewcommand{\baselinestretch}{1.0}
\lstinputlisting{../TireGround/tests/Geometry-test1.cc}
\renewcommand{\baselinestretch}{1.25}
%
\subsection{Geometry-test2.cc}
\renewcommand{\baselinestretch}{1.0}
\lstinputlisting{../TireGround/tests/Geometry-test2.cc}
\renewcommand{\baselinestretch}{1.25}
%
\subsection{Geometry-test3.cc}
\renewcommand{\baselinestretch}{1.0}
\lstinputlisting{../TireGround/tests/Geometry-test3.cc}
\renewcommand{\baselinestretch}{1.25}
%
\subsection{Geometry-test4.cc}
\renewcommand{\baselinestretch}{1.0}
\lstinputlisting{../TireGround/tests/Geometry-test4.cc}
\renewcommand{\baselinestretch}{1.25}
%
\section{Tests per il Modello Magic Formula}
%
\subsection{MagicFormula-test1.cc}
\renewcommand{\baselinestretch}{1.0}
\lstinputlisting{../TireGround/tests/MagicFormula-test1.cc}
\renewcommand{\baselinestretch}{1.25}
%
\subsection{MagicFormula-test2.cc}
\renewcommand{\baselinestretch}{1.0}
\lstinputlisting{../TireGround/tests/MagicFormula-test2.cc}
\renewcommand{\baselinestretch}{1.25}

\backmatter

\chapterstyle{default} % Reset the chapter style back to the default used for non-content chapters

%----------------------------------------------------------------------------------------
%	BIBLIOGRAPHY
%----------------------------------------------------------------------------------------

\printbibliography

%----------------------------------------------------------------------------------------
%	INDEX
%----------------------------------------------------------------------------------------

\printindex % Print the index

%----------------------------------------------------------------------------------------

\end{document}
