\chapter{Introduzione}
\label{Introduzione}
%
\section{Obiettivi}
Il presente lavoro di tesi ha preso avvio dalla collaborazione tra il Dipartimento di Ingegneria Industriale dell'Università degli Studi di Trento e AnteMotion S.r.l., azienda specializzata in realtà virtuale e simulazione \textit{multibody} nel campo \textit{automotive}. In particolare, il modello di veicolo e pneumatico precedentemente studiati da \citeauthor{Larcher} in \cite{Larcher} saranno integrati nel simulatore di guida in tempo reale di AnteMotion. Pertanto, lo sviluppo dei modelli è stato finalizzato a minimizzare i tempi di compilazione massimizzando invece l'accuratezza. La necessità di sviluppare un algoritmo che calcoli i parametri dell'interazione tra terreno (rappresentato con una \textit{mesh} triangolare) e pneumatico (rappresentato come uno o più dischi indeformabili) getta le basi per il lavoro svolto.
%
\section{Il problema}
La simulazione risolve alcuni dei problemi relativi al mondo della progettazione in modo sicuro ed efficiente, senza la necessità di costruire un prototipo dell'oggetto fisico. A differenza della modellazione fisica, che può coinvolgere il sistema reale o una copia in scala di esso, la simulazione è basata sulla tecnologia digitale e utilizza algoritmi ed equazioni per rappresentare il mondo reale al fine di imitare l'esperimento. Ciò comporta diversi vantaggi in termini di tempo, costi e sicurezza. Infatti, il modello digitale può essere facilmente riconfigurato e analizzato, mentre questo è solitamente impossibile o troppo oneroso del punto di vista di tempi e/o costi da fare con il sistema reale \cite{Anu}.

Al giorno d'oggi esistono numerosi modelli di veicolo e pneumatico. Certamente, più semplice è il modello più veloce è la risoluzione delle equazioni costituenti, quindi, a seconda delle applicazioni, dev'essere scelto il modello con la giusta complessità. Per la maggior parte delle applicazioni di guida autonoma, un modello semplice è adeguato per caratterizzare con un livello di dettaglio sufficiente il comportamento del veicolo, e poiché queste analisi sono molto spesso fatte con l'ausilio di \ac{HIL}, il modello dinamico del veicolo dev'essere risolto in tempo reale con tipico passo di tempo di un millisecondo. Il vincolo di esecuzione in tempo reale implica la scelta un modello di veicolo che sia velocemente risolvibile, ciò significa che i modelli semplici con pochi parametri, di solito modelli lineari a due ruote, sono particolarmente adatti per questo tipo di applicazioni. Tuttavia, ci sono alcune situazioni che richiedono modelli più dettagliati, come ad esempio l'azione prodotta da un \ac{ADAS}, ovvero una manovra di sicurezza come l'elusione degli ostacoli o una frenata di emergenza, poiché il veicolo è spinto nella maggior parte dei casi al limite delle sue prestazioni \cite{impacts}. In queste condizioni di guida si devono tenere conto di molti fattori come ad esempio il comportamento degli pneumatici che, spostandosi nella regione non lineare, fa sì che i fenomeni transitori non siano più trascurabili. Questo implica la necessità di utilizzare un modello più dettagliato di quello utilizzato per la guida in condizioni \textit{standard}.

L'accuratezza dinamica del modello è di grande rilevanza per ricavare previsioni realistiche delle prestazioni del veicolo e del sistema di controllo. È importante notare che modellare in modo esaustivo tutti i sistemi di un'auto sarebbe un compito estremamente arduo e a talvolta anche impossibile. Esistono quindi modelli empirici come il modello della \textit{Magic Formula} di Hans Pacejka, che cerca di imitare il reale comportamento del sistema. Il calcolo dei parametri di questo tipo di modelli richiede l'interpolazione di un insieme di dati di grandi dimensioni, e può quindi essere numericamente inefficiente o comunque troppo oneroso in termini di tempo.

Lo scopo di questo lavoro si collega a quello già svolto da \citeauthor{Larcher} in \cite{Larcher} in cui, grazie a un modello di veicolo completo con 14 gradi di libertà ha fornito un modello in grado di catturare con un livello di dettaglio appropriato il comportamento del veicolo quando viene spinto alle massime prestazioni. La necessità di calcolare in tempo reale i parametri di input per il modello di ruota scelto da \cite{Larcher} definisce l'obiettivo di questo lavoro. In particolare lo scopo è quello di implementare una libreria in linguaggio \texttt{C++} che con alcuni semplici parametri in \textit{input} come la denominazione \ac{ETRTO} e la posizione nello spazio dello pneumatico, calcola i dati relativi all'interazione dello stesso con strada cercando di minimizzare i tempi di compilazione.
