\chapter{Introduzione}
\label{Introduzione}
%
\section{Obiettivi}
La motivazione di questa tesi sta nella trovata collaborazione tra il \textit{Dipartimento di Ingegneria Industriale} dell'Università di Trento e \textit{AnteMotion S.r.l.}, azienda specializzata in realtà virtuale e simulazione di veicoli multibody. In particolare il modello di veicolo e pnumatico precedentemente studiati da \citeauthor{Larcher} in \cite{Larcher} sarà integrato nel simulatore di guida in tempo reale di AnteMotion. Pertanto, lo sviluppo del modello è stato finalizzata a minimizzare i tempi di compilazione massimizzando l'accuratezza del modello. La necessità di sviluppare un algoritmo che calcoli i parametri dell'interazione tra terreno e pneumatico getta le basi per il lavoro svolto.
%
\section{Il problema}
La \textit{simulazione} risolve alcuni dei problemi relativi al mondo della progettazione in modo sicuro ed efficiente, senza la necessità di costruire un prototipo dell'oggetto fisico. A differenza della modellazione fisica, che può coinvolgere il sistema reale o una copia in scala di esso, la simulazione è basata sulla tecnologia digitale e utilizza algoritmi ed equazioni per rappresentare il mondo reale al fine di imitare l'esperimento reale. Ciò comporta diversi vantaggi in termini di tempo, costi e sicurezza. Infatti, il modello digitale può essere facilmente riconfigurato e analizzato, mentre questo è solitamente impossibile o troppo oneroso del punto di vista di tempi e costi da fare con il sistema reale \cite{Anu}. Al giorno d'oggi esistono numerosi modelli di veicolo e pneumatico, certamente, più semplice è il modello, più veloce è la risoluzione delle equazioni costituenti e, a seconda delle applicazioni, deve essere scelta la giusta complessità per il modello. Per la maggior parte delle applicazioni di guida autonoma, un modello semplice è sufficiente per caratterizzare con un livello di dettaglio sufficiente il comportamento del veicolo, e poiché queste analisi sono molto spesso fatte con l'ausilio di \ac{HIL}, il modello dinamico del veicolo deve essere risolto in tempo reale con tipico passo di tempo di 1 millisecondo. Il vincolo in tempo reale implica un modello di veicolo di calcolo veloce, ciò significa che i modelli semplici con pochi parametri, di solito modelli lineari a singola traccia, sono particolarmente adatti per questo tipo di applicazioni. Tuttavia ci sono alcune situazioni che richiedono modelli più dettagliati, come ad esempio l'azione prodotta da un \ac{ADAS}, ovvero una manovra di sicurezza come l'elusione degli ostacoli o una frenata di emergenza, poiché il veicolo è spinto nella maggior parte dei casi al limite delle sue prestazioni \cite{impacts}. In queste condizioni di guida si devono tenere conto di molti fattori come ad esempio il comportamento degli pneumatici, che si sposta nella regione non lineare e i fenomeni transitori non sono più trascurabili. Ciò significa che un modello più dettagliato di quello utilizzato per la guida in condizioni "standard". L'accuratezza dinamica del modello è di grande importanza per ricavare previsioni realistiche delle prestazioni del veicolo e del sistema di controllo. È importante notare che modellare in modo esaustivo tutti i sistemi di un'auto sarebbe un compito estremamente arduo e a volte anche impossibile. Esistono quindi modelli empirici come il modello della \textit{MagicFormula} di Hans Pacejka e il modello \textit{Fiala} che cercano di imitare il reale comportamento del sistema. Il calcolo dei parametri di questo tipo di modelli richiede l'interpolazione di un set di dati di grandi dimensioni, e può quindi essere numericamente inefficiente o comunque troppo oneroso in termini di tempo.\\
Per studiare il comportamento del sistema in diversi scenari di guida, i moderni strumenti di simulazione spesso richiedono una grande quantità di di dati e l'unico modo per ottenerli è esecuzione stessa di migliaia di simulazioni. A questo proposito è quindi necessario un conducente virtuale o artificiale in grado di controllare il veicolo fino ai limiti di guidabilità. Pertanto, la caratteristica principale di un driver artificiale o virtuale è la capacità di guidare una varietà di veicoli con diverse caratteristiche dinamiche. Indipendentemente dall'architettura e dalla metodologia utilizzata per sviluppare il conducente artificiale, deve utilizzare una sorta di modello dinamico che rappresenta il comportamento del veicolo controllato. Si tratta dunque di una sorta di modello di comportamento dinamico del veicolo.\\
Lo scopo di questo lavoro si collega a quello già svolto da \citeauthor{Larcher} in \cite{Larcher}, dove grazie a un modello di veicolo completo con 14 gradi di libertà ha fornito un modello in grado di catturare con un livello di dettaglio appropriato il comportamento del veicolo quando viene spinto ai suoi limiti di maneggevolezza. La necessità di calcolare in tempo reale i parametri di input per il modello di ruota scelto da \cite{Larcher} definisce l'obiettivo di questo lavoro. Ovvero di avere una libreria scritta in \texttt{C++}, che con alcuni semplici parametri in input come la denominazione \ac{ETRTO} dello pneumatico e la posizione nello spazio, calcola i dati relativi all'interazione pneumatico strada quali l'intersezione del punto sotto il centro ruota, l'area di contatto, e l'inclinazione locale del piano strada. Il tutto cercando di minimizzare i tempi di compilazione.
%
\cite{Moller} \cite{RayTriangle} \cite{Swift} \cite{Schmeitz}