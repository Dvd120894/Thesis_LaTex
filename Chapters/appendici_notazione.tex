\chapter{Convenzioni e Notazioni}
\label{Notazioni}
%
\section{Convenzioni}
La convenzione utilizzata per definire gli assi del sistema di riferimento dello pneumatico è la \ac{ISO} 8855.
%
% INSERIRE FIGURA
%
\section{Matrice di Trasformazione}
Per descrivere sia l'orientamento che la posizione di un sistema di assi nello spazio, viene introdotta la \textit{matrice roto-traslazione}, chiamata anche \textit{matrice di trasformazione}. Questa notazione permette di impiegare le operazioni matrice-vettore per l'analisi di posizione, velocità e accelerazione. La forma generale di una matrice di trasformazione è del tipo:
%
\begin{equation}
T_m = \left[
\begin{array}{ccc|c}
& & & O_{mx}\\
\multicolumn{3}{c|}{\multirow{3}{*}{\raisebox{20mm}{\scalebox{1.5}{$[R_m]$}}}} & O_{my}\\
 & & & O_{mz}\\ \hline
0 & 0 & 0 & 1
\end{array}\right]
\end{equation}\\
%
dove $R_m$ è la matrice di rotazione $3 \times 3$ del sistema di riferimento in movimento e $O_{mx}$, $O_{my}$ e $O_{mz}$ sono le coordinate della sua origine nel sistema di riferimento assoluto o nativo. L'introduzione dell'elemento fittizio 1 nel vettore della posizione di origine e la successiva spaziatura interna zero della matrice rende possibili le moltiplicazioni matrice-vettore, rendendo la matrice di trasformazione un modo compatto e conveniente per la descrizione dei sistemi di riferimento. Si noti che per i vettori, le informazioni traslazionali vengono trascurate imponendo l'elemento fittizio pari a 0.