\chapter{Codice della Libreria C++}
\label{LibraryCode}
%
\section{TireGround.hh}
\footnotesize
\renewcommand{\baselinestretch}{1.0}
\lstinputlisting[language=C++]{../TireGround/include/TireGround.hh}
\renewcommand{\baselinestretch}{1.25}
%
\section{RoadRDF.hh}
\renewcommand{\baselinestretch}{1.0}
\lstinputlisting[language=C++]{../TireGround/include/RoadRDF.hh}
\renewcommand{\baselinestretch}{1.25}
%
\section{RoadRDF.cc}
\renewcommand{\baselinestretch}{1.0}
\lstinputlisting[language=C++]{../TireGround/src/RoadRDF.cc}
\renewcommand{\baselinestretch}{1.25}
%
\section{PatchTire.hh}
\renewcommand{\baselinestretch}{1.0}
\lstinputlisting[language=C++]{../TireGround/include/PatchTire.hh}
\renewcommand{\baselinestretch}{1.25}
%
\section{PatchTire.cc}
\renewcommand{\baselinestretch}{1.0}
\lstinputlisting[language=C++]{../TireGround/src/PatchTire.cc}
\renewcommand{\baselinestretch}{1.25}
%
\chapter{Codice dei Tests}
\label{TestsCode}
%
\section{Tests Geometrici}
%
\subsection{Geometry-test1.cc}
\renewcommand{\baselinestretch}{1.0}
\lstinputlisting{../TireGround/tests/Geometry-test1.cc}
\renewcommand{\baselinestretch}{1.25}
%
\subsection{Geometry-test2.cc}
\renewcommand{\baselinestretch}{1.0}
\lstinputlisting{../TireGround/tests/Geometry-test2.cc}
\renewcommand{\baselinestretch}{1.25}
%
\subsection{Geometry-test3.cc}
\renewcommand{\baselinestretch}{1.0}
\lstinputlisting{../TireGround/tests/Geometry-test3.cc}
\renewcommand{\baselinestretch}{1.25}
%
\subsection{Geometry-test4.cc}
\renewcommand{\baselinestretch}{1.0}
\lstinputlisting{../TireGround/tests/Geometry-test4.cc}
\renewcommand{\baselinestretch}{1.25}
%
\section{Tests per il Modello Magic Formula}
%
\subsection{MagicFormula-test1.cc}
\renewcommand{\baselinestretch}{1.0}
\lstinputlisting{../TireGround/tests/MagicFormula-test1.cc}
\renewcommand{\baselinestretch}{1.25}
%
\subsection{MagicFormula-test2.cc}
\renewcommand{\baselinestretch}{1.0}
\lstinputlisting{../TireGround/tests/MagicFormula-test2.cc}
\renewcommand{\baselinestretch}{1.25}