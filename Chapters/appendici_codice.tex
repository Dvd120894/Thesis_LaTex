\chapter{Convenzioni e notazioni}
\label{Notazioni}
%
\subsection{Sistemi di riferimento}
La convenzione utilizzata per definire gli assi del sistema di riferimento della vettura è la \ac{ISO} 8855.

\begin{figure}[h!]
	\centering
	\includegraphics[width=0.8\linewidth]{Figures/iso_convention}
	\caption{Rappresentazione degli assi del sistema di riferimento della vettura secondo la convenzione ISO-V.}
	Da: \citeauthor{Ginebra}, \citetitle{Ginebra}.
	\label{isoconventionv}
\end{figure}

\noindent
Il sistema di riferimento della ruota è conforme alla convenzione \ac{ISO}-V, la cui disposizione degli assi è illustrata nella \figurename{ \ref{isoconventionc}}. L'origine del sistema di riferimento del vettore ruota è posta in corrispondenza del centro della ruota mentre posizione e orientamento relativi rispetto al sistema di riferimento del telaio sono definiti attraverso il modello della sospensione descritto in \cite{Larcher}.

\begin{figure}[h!]
	\centering
	\includegraphics[width=0.7\linewidth]{Figures/iso_convention_wheel}
	\caption{Rappresentazione degli assi  del sistema di riferimento dello pneumatico secondo la convenzione ISO-C.}
	Da: Documentazione \texttt{MFeval}.
	\label{isoconventionc}
\end{figure}
%
\subsection{Matrice di trasformazione}
Per descrivere sia l'orientamento che la posizione di un sistema di assi nello spazio, viene introdotta la matrice roto-traslazione, chiamata anche matrice di trasformazione. Questa notazione permette di impiegare le operazioni matrice-vettore per l'analisi di posizione, velocità e accelerazione. La forma generale di una matrice di trasformazione è del tipo:
%
\begin{equation}
T_m = \left[
\begin{array}{ccc|c}
& & & O_{mx}\\
\multicolumn{3}{c|}{\multirow{3}{*}{\raisebox{20mm}{\scalebox{1.5}{$[R_m]$}}}} & O_{my}\\
& & & O_{mz}\\ \hline
0 & 0 & 0 & 1
\end{array}\right]
\end{equation}\\
%
dove $R_m$ è la matrice di rotazione $3 \times 3$ del sistema di riferimento in movimento e $O_{mx}$, $O_{my}$ e $O_{mz}$ sono le coordinate della sua origine nel sistema di riferimento assoluto o nativo.

L'introduzione dell'elemento fittizio 1 nel vettore della posizione di origine e la successiva spaziatura interna zero della matrice rende possibili le moltiplicazioni matrice-vettore, rendendo la matrice di trasformazione una notazione compatta e conveniente per la descrizione dei sistemi di riferimento. Si noti che per i vettori, le informazioni traslazionali vengono trascurate imponendo l'elemento fittizio pari a 0.
%
\chapter{Documentazione della libreria \texttt{TireGround}}
\label{Documentation}
%
\texttt{Doxygen} è un \textit{software} comunemente utilizzato per generare documentazione direttamente dalle annotazioni nei \textit{file} \texttt{C++}. Questo \textit{tool} supporta anche altri linguaggi di programmazione popolari come \texttt{C}, \texttt{Objective-C}, \texttt{C\#}, \texttt{PHP},  \texttt{Java},  \texttt{Python}, \texttt{Fortran},  \texttt{VHDL},  \texttt{Tcl} e in una certa misura  \texttt{D}.

\texttt{Doxygen} può essere utile per i seguenti motivi.
\begin{itemize}
	\item Può generare una documentazione da utilizzare \textit{online} (in \texttt{HTML}) e/o un manuale di riferimento \textit{offline} (in \LaTeX) da una serie di \textit{file} sorgente opportunamente annotati. C'è anche il supporto per generare \textit{output} in \texttt{RTF} (MicroSoft Word), \texttt{PostScript}, \texttt{PDF} con \textit{hyperlink} e \texttt{HTML} compresso. La documentazione viene estratta direttamente dalle fonti, il che rende molto più semplice mantenere la documentazione coerente con il codice sorgente.
	\item È possibile configurare \texttt{doxygen} per estrarre la struttura del codice da \textit{file} sorgente non documentati. Questo è molto utile per analizzare rapidamente ed efficacemente i \textit{file} sorgente di grandi dimensioni. Doxygen può anche visualizzare le relazioni tra i vari elementi mediante grafici di dipendenza, diagrammi di ereditarietà e diagrammi di collaborazione, tutti generati automaticamente.
\end{itemize}
\texttt{Doxygen} è sviluppato su Mac OS X e Linux, ma è configurato per essere altamente portabile. Di conseguenza, funziona anche con la maggior parte degli altri sistemi Unix. Inoltre, sono disponibili eseguibili per Windows.

\includepdf[pages=-, offset=10mm 0mm]{refman.pdf}

\chapter{Codice sorgente delle prove numeriche}
\label{TestsCode}
%
\section{\textit{Tests} di tipo geometrico}
%
\subsection{\texttt{Geometry-test1.cc}}
\renewcommand{\baselinestretch}{1.0}
\lstinputlisting{../TireGround/tests/Geometry-test1.cc}
\renewcommand{\baselinestretch}{1.25}
%
\subsection{\texttt{Geometry-test2.cc}}
\renewcommand{\baselinestretch}{1.0}
\lstinputlisting{../TireGround/tests/Geometry-test2.cc}
\renewcommand{\baselinestretch}{1.25}
%
\subsection{\texttt{Geometry-test3.cc}}
\renewcommand{\baselinestretch}{1.0}
\lstinputlisting{../TireGround/tests/Geometry-test3.cc}
\renewcommand{\baselinestretch}{1.25}
%
\subsection{\texttt{Geometry-test4.cc}}
\renewcommand{\baselinestretch}{1.0}
\lstinputlisting{../TireGround/tests/Geometry-test4.cc}
\renewcommand{\baselinestretch}{1.25}
%
\section{\textit{Tests} per il modello a singolo disco}
%
\subsection{texttt{MagicFormula-test1.cc}}
\renewcommand{\baselinestretch}{1.0}
\lstinputlisting{../TireGround/tests/MagicFormula-test1.cc}
\renewcommand{\baselinestretch}{1.25}
%
\subsection{texttt{MagicFormula-test2.cc}}
\renewcommand{\baselinestretch}{1.0}
\lstinputlisting{../TireGround/tests/MagicFormula-test2.cc}
\renewcommand{\baselinestretch}{1.25}
%
\section{Tests per il modello a più dischi}
%
\subsection{texttt{MultiDisk-test1.cc}}
\renewcommand{\baselinestretch}{1.0}
\lstinputlisting{../TireGround/tests/MultiDisk-test1.cc}
\renewcommand{\baselinestretch}{1.25}
%
\subsection{texttt{MultiDisk-test2.cc}}
\renewcommand{\baselinestretch}{1.0}
\lstinputlisting{../TireGround/tests/MultiDisk-test2.cc}
\renewcommand{\baselinestretch}{1.25}