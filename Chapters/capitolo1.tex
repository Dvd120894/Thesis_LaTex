\chapter{Il Pneumatico}
\label{Pneumatico}
%
\section{Una Breve Introduzione}
Gli pneumatici sono probabilmente i componenti più complessi di un'auto in quanto combinano decine di componenti che devono essere formati, assemblati e curati insieme. Il loro successo finale dipende dalla loro capacità di fondere tutti i componenti separati in un prodotto coeso che soddisfa le esigenze del conducente \cite{rill}. Gli pneumatici sono caratterizzati da un comportamento altamente non lineare con una dipendenza da diversi fattori costruttivi e ambientali. Tuttavia, le forze di contatto possono essere descritte completamente da un vettore di forza risultante applicato in un punto specifico della patch di contatto e da un vettore di coppia, come illustrato nella \figurename\ref{label}.
%
% INSERIRE FIGURA 2.6
%
Componenti cruciali per la movimentazione dei veicoli e il comportamento di guida, le forze degli pneumatici richiedono particolare attenzione e cura soprattutto quando, lungo il comportamento stazionario, il comportamento non stazionario deve essere coperto. Attualmente, è possibile identificare tre gruppi di modelli:
%
\begin{itemize}
	\item modelli matematici;
	\item modelli fisici;
	\item combinazione dei precedenti.
\end{itemize}
%
La prima tipologia di modello tenta di rappresentare le caratteristiche fisiche del pneumatico attraverso una descrizione puramente matematica. Pertanto questi tipi di modelli partono da un curve caratteristiche ricavate sperimentalmente e cercano di derivare un comportamento approssimativo dall'interpolazione di dati. Un esempio ben noto di questo approccio è il modello di Pacejka o \textit{Magic Formula Tire Model} \cite{hans}. Questo tipo di modellazione è adatta per la simulazione di manovre di guida, dove il comportamento di interesse è per lo più la gestione del veicolo e le frequenze di uscita sono ben al di sotto delle frequenze di risonanza della cintura dello pneumatici. I modelli fisici o i modelli ad alta frequenza, come i modelli agli elementi finiti, sono in grado di rilevare fenomeni di frequenza più elevata come le vibrazioni della membrana. Ciò permette di valutare il comfort di guida di un veicolo. Dal punto di vista del calcolo, i modelli fisici complessi richiedono molto tempo al computer per essere risolto, nonché molti dati. Dall'altro lato, i modelli matematici sono veloci in termini di calcolo, ma richiedono un'accurata pre-elaborazione dei dati sperimentali. La terza tipologia di modelli consiste in un'estensione dei modelli matematici attraverso le leggi fisiche al fine di coprire una gamma di frequenza più ampia.\\
Il modello di pneumatico sviluppato nel modello di veicolo presentato da \citeauthor{Larcher} in \cite{Larcher} si basa sulla Magic Formula 6.2. Una panoramica generale sui modelli della Magic Formula e sul calcolo dello slip degli pneumatici è descritta nelle sezioni seguenti, mentre l'insieme completo delle equazioni del modello implementato è riportato nell'Appendice.
Come vedremo nel Capitolo 3 il modello di pneumatico deve essere combinato con un'interfaccia di pneumatico/strada modellata adeguata per ottenere risultati significativi.
%
\section{Il modello Pacejka}
Uno dei modelli di pneumatici più utilizzati è il cosiddetto \textit{Magic-Formula Model} sviluppato da \citeauthor{bakker} in \cite{bakker}. Questo modello è stato poi rivisto e l'ultima versione è riportata in \cite{hans}. Il Magic-Formula Model consiste in una pura descrizione matematica del rapporto input-output del contatto pneumatico-strada. Questa formulazione collega le variabili di forza con lo slip rigido del corpo che vengono trattati nelle sezioni successive. La forma generale della funzione di descrizione può essere scritta come:
\begin{equation}
y(x) = D\sin\{C\arctan[B(x + S_h ) - E(B(x + S_h ) - \arctan(B(x + S_h )))]\} + S_v
\end{equation}



\begin{table}[h]
	\centering
	\begin{tabular}{cl}
		\toprule \emph{Parametri} & \textsc{Significato}\\
		\midrule
		$B$ & Fattore di rigidezza \\
		$C$ & Fattore di forma \\
		$D$ & Valore massimo della forza o coppia \\
		$E$ & Fattore di curvatura \\
		$S_v$ & Spostamento in verticale della curva caratteristica \\
		$S_h$ & Spostamento in orizzontale della curva caratteristica \\
		\bottomrule
	\end{tabular}
	\caption{Significato dei valori del modello di Pacejka.}
	\label{MFParams}
\end{table}

%
\section{Standardizzazione ETRTO}

%
\section{Contatto con la Superficie Stradale}
La posizione e l'orientamento della ruota in relazione al sistema fissato a terra sono dati dal telaio di riferimento del vettore ruota $RF_{wh_{i}}$ che viene calcolato risolvendo le equazioni dinamiche del sistema ottenuto nel Capitolo 2 in \cite{Larcher} istante per istante. Supponendo che il profilo stradale potrebbe essere rappresentato da una funzione arbitraria di due coordinate spaziali 
%
\begin{equation}
z=z(x,y)
\end{equation}
%
su una irregolare, il punto di contatto $P$ non può essere calcolato direttamente. Così, come prima approssimazione siamo in grado di identificare un punto $P$, che è definito come una semplice traslazione del centro ruota $M$:
%
\begin{equation}
P^\star = M-R_0\textbf{\textit{e}}_{zC}
\begin{bmatrix}
x^\star\\
y^\star\\
z^\star
\end{bmatrix}
\end{equation}
%
dove $R_0$ è il raggio dello pneumatico indeformato e $\textbf{\textit{e}}_{zC}$ è il vettore unitario che definisce l'asse $z_c$ del sistema di riferimento del vettore ruota.\\
$
% INSERIRE FIGURA
$
La prima stima del sistema di riferimento del punto di contatto $RF_{PC^\star}$ è un frame con origine in $P^\star$ e orientamento dell'asse definito dall'orientamento dell'asse del sistema portante della ruota.
%
\begin{equation}
RF_{PC^\star} = \left[
\begin{array}{ccc|c}
& & & x^\star\\
\multicolumn{3}{c|}{\multirow{3}{*}{\raisebox{20mm}{\scalebox{1.5}{$[R_{RF_{wh}}]$}}}} & y^\star\\
& & & z^\star\\ \hline
0 & ~~0 & 0 & 1
\end{array}\right]
\end{equation}\\
%
Ora, i vettori di unità $\textbf{\textit{e}}_x$ ed $\textbf{\textit{e}}_y$, che descrivono il piano locale nel punto $P$, possono essere ottenuti dalle seguenti equazioni:
%
\begin{equation}
\textbf{\textit{e}}_x = \frac{\textbf{\textit{e}}_{yC}\times\textbf{\textit{e}}_{n}}{|\textbf{\textit{e}}_{yC}\times\textbf{\textit{e}}_{n}|}
\qquad
\textbf{\textit{e}}_{y} = \textbf{\textit{e}}_{n}\times\textbf{\textit{e}}_{x}
\end{equation}
%
Al fine di ottenere una buona approssimazione del piano pista locale in termini di inclinazione longitudinale e laterale, sono stati utilizzati un insieme di quattro punti di campionamento $(Q^\star_1, Q^\star_2, Q^\star_3, Q^\star_4)$ che sono rappresentati graficamente in \figurename\ref{Fig}. I punti di campionamento sono definiti sul piano locale del punto di contatto $RF_{PC^\star}$, poiché lo spostamento longitudinale e laterale dall'origine del sistema, che è $P^\star$. I vettori di spostamento sono definiti come:
%
\begin{equation}
\begin{split}
_{PC^\star}\textbf{\textit{r}}_{Q^\star_{1,2}} = \pm \Delta x \\
_{PC^\star}\textbf{\textit{r}}_{Q^\star_{3,4}} = \pm \Delta y
\end{split}
\end{equation}
%
e quindi, i quattro punti di campionamento sono:
%
\begin{equation}
\begin{split}
_{P^\star}\textbf{\textit{r}}_{Q^\star_{1,2}} = P^\star \pm \Delta x \textbf{\textit{e}}_{xPC^\star} \\
_{P^\star}\textbf{\textit{r}}_{Q^\star_{3,4}} = P^\star \pm \Delta y \textbf{\textit{e}}_{yPC^\star}
\end{split}
\end{equation}
%
Al fine di campionare la patch di contatto nel modo più efficiente possibile, le distanze di $\Delta x$ e $\Delta y$, dell'equazione precedente, vengono regolate in base al raggio del pneumatico indeformato $R_0$ e alla larghezza del pneumatico $B$. I valori di queste due quantità possono essere trovate in letteratura e sono $\Delta x = 0.1 R_0$ e $\Delta x = 0.3 B$. Attraverso questa definizione, si può ottenere un comportamento realistico durante la simulazione.
%
% METTERE IMAGINE
%
Ora il componente traccia $z$ in corrispondenza dei quattro punti campione viene valutato attraverso la funzione $z(x,y)$ (Eq. 3.1), quindi, aggiornando la terza coordinata dei punti di sondaggio $Q^\star_i$, otteniamo i corrispondenti punti campione $Q_i$ sulla superficie della pista locale. La linea fissata dai punti $Q_1$, $Q_2$ e rispettivamente $Q_3$, $Q_4$, può ora essere utilizzata per definire il vettore normale del piano della pista locale (\figurename\ref{label}). Pertanto, il vettore normale è definito come:
%
\begin{equation}
\textbf{\textit{e}}_n = \frac{\textbf{\textit{r}}_{Q_1 Q_2} \times \textbf{\textit{r}}_{Q_4 Q_3}}{|\textbf{\textit{r}}_{Q_1 Q_2} \times \textbf{\textit{r}}_{Q_4 Q_3}|}
\end{equation}
%
dove sono $\textbf{\textit{r}}_{Q_2 Q_1}$ e $\textbf{\textit{r}}_{Q_4 Q_3}$ sono i vettori che puntano rispettivamente da $Q_1$ a $Q_2$ e da $Q_3$ a $Q_4$. Applicando Eq 3.4 è ora possibile calcolare i vettori unitari $\textbf{\textit{e}}_{x}$ e $\textbf{\textit{e}}_{y}$ del piano di locale del punto di contatto. Il punto di contatto $P$ si ottiene aggiornando le coordinate del primo punto di prova $P^\star$, con il valore medio delle tre coordinate spaziali dei quattro punti campione.
%
\begin{equation}
P = \frac{1}{4}\begin{bmatrix}
\sum_{i=1}^{4} x_i \\
\sum_{i=1}^{4} y_i \\
\sum_{i=1}^{4} z_i
\end{bmatrix}
\end{equation}
%
Infine possiamo mettere assieme tutte le componenti del piano di riferimento del punto di contatto finale ottenendo:
%
\begin{equation}
RF_{PC} = \left[
\begin{array}{ccc|c}
& & & x_P\\
\multirow{3}{*}{\raisebox{20mm}{\scalebox{1.5}{$\left[\textbf{\textit{e}}_x\right]$}}} & \multirow{3}{*}{\raisebox{20mm}{\scalebox{1.5}{$\left[\textbf{\textit{e}}_y\right]$}}} & \multirow{3}{*}{\raisebox{20mm}{\scalebox{1.5}{$\left[\textbf{\textit{e}}_z\right]$}}} & y_P\\
& & & z_P\\ \hline
0 & 0 & 0 & 1
\end{array}\right]
\end{equation}\\
Attraverso questo approccio di modellazione, le informazioni della traccia locale normal vector $\textbf{\textit{e}}_{n}$, insieme al punto di contatto locale $P$ sono in grado di rappresentare l'irregolarità locale in modo soddisfacente. Come accade in realtà, bordi taglienti o discontinuità del manto stradale saranno smussate da questo approccio. Alcuni casi dimostrativi sono illustrati nella \figurename\ref{ssf}.
