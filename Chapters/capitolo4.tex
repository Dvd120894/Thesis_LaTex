\chapter{La Libreria \texttt{TireGround}}
\label{Codice}
%
\section{Organizzazione}
La libreria \texttt{TireGround} è stata organizzata in due parti, la prima gestisce la superficie stradale mentre la seconda gestisce i modelli di pneumatico. Si sviluppa all'interno dell'omonimo \texttt{namespace TireGround} nel quale vengono inoltre dichiarati con \texttt{typedef} alcuni tipi che verranno utilizzati nelle due sottosezioni.

Verranno ora riportate le informazioni di maggior rilievo per ognuna delle due parti della libreria.


\subsection{Gestione della superficie stradale} 
La gestione della superficie stradale avviene all'interno del \texttt{namespace RDF}. In quest'ultimo vengono raccolti alcuni tipi dichiarati con \texttt{typedef} presenti solo nel \texttt{namespace RDF}. Lo spazio dei nomi \texttt{RDF} contiene tutti le classi e la funzioni per gestire la \textit{mesh} a partire dal file in formato \ac{RDF}.
%
\paragraph{\texttt{BBox2D}}
Questa classe contiene tutte le informazioni per definire e manipolare una \ac{BB} bidimensionale. Consiste nella descrizione geometrica dell'oggetto \ac{BB}.  I metodi più importanti di questa classe sono i seguenti.
\begin{itemize}
	\item \texttt{clear} $-$ Elimina il dominio della \ac{BB} settando tutti i quattro valori su \texttt{quietNaN}.
	\item \texttt{updateBBox2D} $-$ Aggiorna il dominio della \ac{BB} settando i suoi valori secondo il massimo ingombro dato dai tre vertici nello spazio tridimensionale in \textit{input}.
\end{itemize}
\begin{table}[h!]
	\centering
	\begin{tabular}{|c|c|c|c|c|}
		\hline 
		\textbf{Tipo} & \textbf{Nome} & \textit{\textbf{Getter}} & \textit{\textbf{Setter}} & \textbf{Descrizione} \\ \hline 
		\texttt{real\_type} & \texttt{Xmin} & $\bullet$ & $\bullet$ & $X_{min}$ della \ac{BB} \\ \hline 
		\texttt{real\_type} & \texttt{Ymin} & $\bullet$ & $\bullet$ & $Y_{min}$ della \ac{BB} \\ \hline
		\texttt{real\_type} & \texttt{Xmax} & $\bullet$ & $\bullet$ & $X_{max}$ della \ac{BB} \\ \hline
		\texttt{real\_type} & \texttt{Ymax} & $\bullet$ & $\bullet$ & $Y_{max}$ della \ac{BB} \\ \hline
	\end{tabular}
	\caption{Attributi della classe \texttt{BBox2D}.}
	\label{BBox2D}
\end{table}
%
\paragraph{\texttt{Triangle3D}}
Questa classe contiene tutte le informazioni geometriche per definire e manipolare un triangolo con vertici nello spazio tridimensionale. Consiste nella descrizione geometrica dell'oggetto triangolo. I metodi più importanti di questa classe sono i seguenti.
\begin{itemize}
	\item \texttt{Normal} $-$ Calcola la normale alla faccia del triangolo.
	\item \texttt{intersectRay} $-$ Interseca il triangolo con una data semiretta (detta anche raggio), definita da direzione e punto di partenza, e ne calcola il punto di intersezione.
	\item \texttt{intersectPlane} $-$ Interseca il triangolo con un dato piano, definito da normale e punto noto, e ne calcola i punti di intersezione.
\end{itemize}
\begin{table}[h!]
	\centering
	\begin{tabular}{|c|c|c|c|c|}
		\hline 
		\textbf{Tipo} & \textbf{Nome} & \textit{\textbf{Getter}} & \textit{\textbf{Setter}} & \textbf{Descrizione} \\ \hline 
		\texttt{vec3} & \texttt{Vertices[3]} & $\bullet$ & $\bullet$ & Vertici del triangolo \\ \hline 
		\texttt{vec3} & \texttt{Normal} & $\bullet$ & $\bullet$ & Normale al triangolo \\ \hline 
		\texttt{BBox2D} & \texttt{TriangleBBox} & $\bullet$ & $\bullet$ & \ac{BB} del triangolo \\ \hline
	\end{tabular}
	\caption{Attributi della classe \texttt{Triangle3D}.}
\end{table}
%
\paragraph{\texttt{TriangleRoad}}
Questa classe contiene tutte le informazioni geometriche e non geometriche per definire e manipolare un triangolo con vertici nello spazio tridimensionale rappresentante la superficie stradale. È derivato dalla classe \texttt{Triangle3D} e ha inoltre un attributo che permetter di descrivere il coefficiente di attrito nella faccia (detto anche locale). I metodi più importanti sono ereditati dalla classe \texttt{Triangle3D}.
\begin{table}[h!]
	\centering
	\begin{tabular}{|c|c|c|c|c|}
		\hline 
		\textbf{Tipo} & \textbf{Nome} & \textit{\textbf{Getter}} & \textit{\textbf{Setter}} & \textbf{Descrizione} \\ \hline 
		\texttt{real\_type} & \texttt{Friction} & $\bullet$ & $\bullet$ & Coefficiente di attrito $\mu$ \\ \hline
	\end{tabular}
	\caption{Attributi della classe \texttt{TriangleRoad}.}
\end{table}
%
\paragraph{\texttt{MeshSurface}}
Questa classe contiene il vettore di puntatori di tipo \texttt{std::shared}-\texttt{\_ptr} alle istanze della classe \texttt{TriangleRoad} che vengono create durante la parsificazione del file \ac{RDF}. Inoltre, contiene il vettore di puntatori alle \ac{BB} di tipo \texttt{PtrBBox}, che è necessario per calcolare l'albero \ac{AABB}. Quest'ultimo esiste come ulteriore attributo della classe sotto forma di puntatore \texttt{PtrAABB}. I metodi più importanti di questa classe sono i seguenti.
\begin{itemize}
	\item \texttt{set} $-$ Copia la \textit{mesh}.
	\item \texttt{LoadFile} $-$ Parsifica il file dato come \textit{input} e crea le istanze \texttt{TriangleRoad} che costituiscono la \textit{mesh}.
	\item \texttt{updateIntersection} $-$ Interseca l'albero di tipo \ac{AABB} della \textit{mesh} con un altro albero esterno di tipo \ac{AABB} e ne restituisce il vettore dei puntatori di tipo \texttt{std::shared\_ptr} alle istanze della classe \texttt{TriangleRoad} che vengono intersecate.
\end{itemize}
\begin{table}[h!]
	\centering
	\begin{tabular}{|c|c|c|c|c|}
		\hline 
		\textbf{Tipo} & \textbf{Nome} & \textit{\textbf{Getter}} & \textit{\textbf{Setter}} & \textbf{Descrizione} \\ \hline 
		\texttt{TriangleRoad\_list} & \texttt{Friction} & $\bullet$ & & Vettore dei triangoli \\ \hline
		\texttt{std::vector<PtrBBox>} & \texttt{PtrBBoxVec} & $\bullet$ & & Vettore delle \ac{BB} \\ \hline
		\texttt{PtrAABB} & \texttt{PtrTree} & $\bullet$ & & Albero di tipo \ac{AABB} \\ \hline
	\end{tabular}
	\caption{Attributi della classe \texttt{MeshSurface}.}
\end{table}
%
\subsection{Gestione dei modelli di pneumatico} 
La gestione dei modelli di pneumatico avviene nel \texttt{namespace TireGround}. Quest'ultimo contiene tutti le classi e la funzioni per gestire l'intersezione tra lo pneumatico e la \textit{mesh} a partire dalla conoscenza di quest'ultima, della geometria e della posizione dello pneumatico.
%
\paragraph{\texttt{Disk}}
Questa classe contiene tutte le informazioni geometriche per definire e manipolare un disco nello spazio tridimensionale. Consiste nella descrizione geometrica e nel posizionamento dello spazio delle coordinate tridimensionali dell'oggetto disco (il disco viene rappresentato nel sistema di riferimento dello pneumatico). I metodi più importanti di questa classe sono i seguenti.
\begin{itemize}
	\item \texttt{isPointInside} $-$ Controlla se un punto generico nello spazio bidimensionale, definito dal piano in cui giace lo stesso disco, si trova all'interno o all'esterno della circonferenza.
	\item \texttt{intersectSegment} $-$ Trova i punti di intersezione tra la circonferenza esterna del disco e un segmento bidimensionale, che dev'essere definito nel piano in cui giace lo stesso disco. L'intero di \textit{output} fornisce il numero di punti di intersezione.
	\item \texttt{intersectPlane} $-$ Interseca il disco con un piano definito da normale e punto noto. In \textit{output} fornisce l'entità geometrica creata dall'intersezione sotto forma di punto noto e direzione della retta.
	\item \texttt{contactTriangles} $-$ Funzione in \textit{overloading} che consente di ottenere il versore normale e coefficiente attrito medi ponderati sull'area, nonché l'area di contatto stessa all'interno del singolo disco a partire da una serie di triangoli.
	\item \texttt{contactPlane} $-$ Funzione in \textit{overloading} che consente di ottenere l'area di contatto all'interno del singolo disco dato un piano.
\end{itemize}
\begin{table}[h!]
	\centering
	\begin{tabular}{|c|c|c|c|c|}
		\hline 
		\textbf{Tipo} & \textbf{Nome} & \textit{\textbf{Getter}} & \textit{\textbf{Setter}} & \textbf{Descrizione} \\ \hline 
		\texttt{vec2} & \texttt{OriginXZ} & $\bullet$ & $\bullet$ & Coordinate \textit{XZ} del disco \\ \hline 
		\texttt{real\_type} & \texttt{OffsetY} & $\bullet$ & $\bullet$ & Coordinata $Y$ del disco \\ \hline
		\texttt{real\_type} & \texttt{Radius} & $\bullet$ & $\bullet$ & Raggio del disco \\ \hline
	\end{tabular}
	\caption{Attributi della classe \texttt{Disk}.}
	\label{}
\end{table}
%
\paragraph{\texttt{ETRTO}}
Questa classe contiene tutte le informazioni necessarie per definire geometricamente uno pneumatico secondo la normativa \ac{ETRTO}. Consiste nella descrizione geometrica dell'oggetto pneumatico in termini di larghezza totale e di diametro esterno indeformato. Come visto nel Capitolo \ref{Pneumatico} attraverso la nomenclatura \ac{ETRTO} (e.g. 205/65R16) è infatti possibile risalire a tutte le informazioni geometriche che definiscono, anche se in maniera grossolana, lo pneumatico.
\begin{table}[h!]
	\centering
	\begin{tabular}{|c|c|c|c|c|}
		\hline 
		\textbf{Tipo} & \textbf{Nome} & \textit{\textbf{Getter}} & \textit{\textbf{Setter}} & \textbf{Descrizione} \\ \hline 
		\texttt{real\_type} & \texttt{SectionWidth} & $\bullet$ & $\bullet$ & Larghezza dello pneumatico \\ \hline 
		\texttt{real\_type} & \texttt{AspectRatio} & $\bullet$ & $\bullet$ & Rapporto percentuale $H/W$ \\ \hline
		\texttt{real\_type} & \texttt{RimDiameter} & $\bullet$ & $\bullet$ & Diametro del cerchione\\ \hline
		\texttt{real\_type} & \texttt{SidewallHeight} & $\bullet$ & & Altezza della spalla \\ \hline
		\texttt{real\_type} & \texttt{TireDiameter} & $\bullet$ & & Diametro dello pneumatico\\ \hline
	\end{tabular}
	\caption{Attributi della classe \texttt{ETRTO}.}
	\label{}
\end{table}
%
\paragraph{\texttt{ReferenceFrame}}
Questa classe contiene tutte le informazioni per definire e manipolare una terna di riferimento nello spazio tridimensionale. Consiste nel posizionamento dello spazio del sistema di riferimento. I metodi più importanti di questa classe sono i seguenti.
\begin{itemize}
	\item \texttt{setTotalTransformationMatrix} $-$ Posiziona nello spazio il sistema di riferimento grazie alla matrice di trasformazione $4\times4$ fornita come \textit{input}.
	\item \texttt{getEulerAngleX} $-$ Ottiene l'angolo creato dalla rotazione attorno all'asse $Y$ del sistema di riferimento locale rispetto a quello assoluto (lo stesso della \textit{mesh}). L'angolo viene ottenuto in seguito alla fattorizzazione $R_z(\Omega) R_x(\gamma) R_y(\theta)$ e utilizzando il metodo di Eulero.
	\item \texttt{getEulerAngleY} $-$ Come il metodo \texttt{getEulerAngleX}, ma usato per ottenere l'angolo creato dalla rotazione attorno all'asse $Y$.
	\item \texttt{getEulerAngleZ} $-$ Come il metodo \texttt{getEulerAngleX}, ma usato per il ottenere l'angolo creato dalla rotazione attorno all'asse $Z$.
\end{itemize}
\begin{table}[h!]
	\centering
	\begin{tabular}{|c|c|c|c|c|}
		\hline 
		\textbf{Tipo} & \textbf{Nome} & \textit{\textbf{Getter}} & \textit{\textbf{Setter}} & \textbf{Descrizione} \\ \hline 
		\texttt{vec3} & \texttt{Origin} & $\bullet$ & $\bullet$ & Origine della terna \\ \hline 
		\texttt{mat3} & \texttt{RotationMatrix} & $\bullet$ & $\bullet$ & Matrice di rotazione \\ \hline
	\end{tabular}
	\caption{Attributi della classe \texttt{ReferenceFrame}.}
	\label{}
\end{table}
%
\paragraph{\texttt{Shadow}}
Questa classe serve a rappresentare l'ombra dello pneumatico nello spazio bidimensionale. È molto simile alla \texttt{RDF::BBox2D} precedentemente presentata, ma a differenza di quest'ultima permette di calcola l'albero per oggetti di tipo \ac{AABB} a una sola foglia, relativo alla stessa ombra totale, della parte superiore e della parte inferiore del \ac{BB} tridimensionale che racchiude lo pneumatico nello spazio. I metodi più importanti di questa classe sono i seguenti.
\begin{itemize}
	\item \texttt{clear} $-$ Elimina il dominio dell'ombra settando tutti i suoi valori su \texttt{quietNaN}.
	\item \texttt{update} $-$ Aggiorna il dominio dell'ombra settando tutti i suoi valori secondo il massimo ingombro dato dalla geometria dello pneumatico e dalla sua posizione nello spazio.
\end{itemize}
\begin{table}[h!]
	\centering
	\begin{tabular}{|c|c|c|c|c|}
		\hline 
		\textbf{Tipo} & \textbf{Nome} & \textit{\textbf{Getter}} & \textit{\textbf{Setter}} & \textbf{Descrizione} \\ \hline 
		\texttt{PtrAABB} & \texttt{PtrTree} & $\bullet$ & & Albero \ac{AABB} totale\\ \hline
		\texttt{PtrAABB} & \texttt{PtrTree\_U} & $\bullet$ & & Albero \ac{AABB} parte superiore \\ \hline
		\texttt{PtrAABB} & \texttt{PtrTree\_L} & $\bullet$ & & Albero \ac{AABB} parte inferiore \\ \hline
	\end{tabular}
	\caption{Attributi della classe \texttt{Shadow}.}
	\label{}
\end{table}
%
\paragraph{\texttt{SamplingGrid}}
Questa classe contiene tutti i parametri che riguardano la precisione dei calcoli che verranno effettuati nel calcolo della normale al terreno, punto e area di contatto.
\begin{table}[h!]
	\centering
	\begin{tabular}{|c|c|c|c|c|}
		\hline 
		\textbf{Tipo} & \textbf{Nome} & \textit{\textbf{Getter}} & \textit{\textbf{Setter}} & \textbf{Descrizione} \\ \hline 
		\texttt{int\_type} & \texttt{PointsN} & $\bullet$ & $\bullet$ & N° di punti di campionamento \\ \hline 
		\texttt{int\_type} & \texttt{DisksN} & $\bullet$ & $\bullet$ & N° di dischi \\ \hline 
		\texttt{int\_type} & \texttt{Switch} & $\bullet$ & $\bullet$ & \textit{Threshold} per il tipo contatto \\ \hline
	\end{tabular}
	\caption{Attributi della classe \texttt{SamplingGrid}.}
	\label{}
\end{table}
%
\paragraph{\texttt{Tire}}
Questa classe serve a rappresentare lo pneumatico nelle coordinate dello spazio tridimensionale. Consiste nel punto di giunzione tra la classe \texttt{ETRTO} che definisce la geometria dello pneumatico in condizione di riposo e la classe \texttt{ReferenceFrame} che ne definisce invece la posizione nello spazio. È una classe virtuale in quanto viene definita con alcuni metodi puri virtuali. Questi metodi verranno poi sostituiti con nelle classi figlie.
\begin{table}[h!]
	\centering
	\begin{tabular}{|c|c|c|c|c|}
		\hline 
		\textbf{Tipo} & \textbf{Nome} & \textit{\textbf{Getter}} & \textit{\textbf{Setter}} & \textbf{Descrizione} \\ \hline 
		\texttt{SamplingGrid} & \texttt{Precision} & & & Precisione dei calcoli \\ \hline 
		\texttt{ETRTO} & \texttt{TireGeometry} & & & Geometria \\ \hline 
		\texttt{ReferenceFrame} & \texttt{RF} & $\bullet$ & $\bullet$ & Posizione \\ \hline
		\texttt{int\_type} & \texttt{TirePose} & $\bullet$ & $\bullet$ & \textit{Flag} per la posizione \\ \hline
	\end{tabular}
	\caption{Attributi della classe \texttt{Tire}.}
	\label{}
\end{table}
%
\paragraph{\texttt{MagicFormula} e \texttt{MultiDisk}}
Queste classi calcolano tutti i parametri necessari per valutare il contatto tra pneumatico a disco singolo e terreno attraverso la formula di Pacejka. Il metodo più importante di queste classi è
\begin{itemize}
	\item \texttt{setup} $-$ Consente di riposizionare la ruota all'interno della \textit{mesh}.
	\item \texttt{calculateRelativeCamber} $-$ Calcola il camber relativo.
	\item \texttt{getRho} $-$ Calcola l'affondamento del disco nel piano strada locale.
	\item \texttt{gettArea} $-$ Calcola l'area d'intersezione dei dischi.
\end{itemize}
\begin{table}[h!]
	\centering
	\begin{tabular}{|c|c|c|c|}
		\hline 
		\textbf{Tipo} & \textbf{Nome} & \textit{\textbf{Getter}} & \textbf{Descrizione} \\ \hline 
		\texttt{Disk} & \texttt{SingleDisk} &  & Disco rigido \\ \hline 
		\texttt{vec3} & \texttt{Normal} & $\bullet$ & Versore del piano strada \\ \hline
		\texttt{vec3} & \texttt{MeshPoint} & $\bullet$ & Punto di contatto sulla \textit{mesh} \\ \hline
		\texttt{vec3} & \texttt{MeshPoint} & $\bullet$ & Punto di contatto sul disco \\ \hline
		\texttt{real\_type} & \texttt{Friction} & $\bullet$ & Coefficiente di attrito locale \\ \hline
		\texttt{real\_type} & \texttt{Area} & $\bullet$ & Area d'intersezione \\ \hline
	\end{tabular}
	\caption{Attributi della classe \texttt{MagicFormula}.}
	\label{}
\end{table}
%
\begin{table}[h!]
	\centering
	\begin{tabular}{|c|c|c|c|}
		\hline 
		\textbf{Tipo} & \textbf{Nome} & \textit{\textbf{Getter}} & \textbf{Descrizione} \\ \hline 
		\texttt{Disk} & \texttt{DiskVec} &  & Vettore dei dischi \\ \hline 
		\texttt{vec3} & \texttt{NormalVec} & $\bullet$ & Vettore dei versori normali \\ \hline
		\texttt{vec3} & \texttt{MeshPointVec} & $\bullet$ & Vettore dei punti di contatto sulla \textit{mesh} \\ \hline
		\texttt{vec3} & \texttt{DiskPointVec} & $\bullet$ & Vettore dei punti di contatto sul disco \\ \hline
		\texttt{real\_type} & \texttt{FrictionVec} & $\bullet$ & Vettore dei coefficienti di attrito locale \\ \hline
		\texttt{real\_type} & \texttt{AreaVec} & $\bullet$ & Vettore delle aree d'intersezione \\ \hline
		\texttt{vec3} & \texttt{Normal} & $\bullet$ & Versore del piano strada \\ \hline
		\texttt{vec3} & \texttt{MeshPoint} & $\bullet$ & Punto di contatto singolo sulla \textit{mesh} \\ \hline
		\texttt{vec3} & \texttt{MeshPoint} & $\bullet$ & Punto di contatto singolo sul disco \\ \hline
		\texttt{real\_type} & \texttt{Friction} & $\bullet$ & Coefficiente di attrito locale \\ \hline
		\texttt{real\_type} & \texttt{Area} & $\bullet$ & Area totale d'intersezione \\ \hline
	\end{tabular}
	\caption{Attributi della classe \texttt{MultiDisk}.}
	\label{}
\end{table}
%
\section{Librerie Esterne}
Oltre al codice appena descritto sono state utilizzate anche altre due librerie esterne al fine di velocizzare il processo di sviluppo e al contempo di utilizzare una solida base per le operazione più complesse, ovvero le operazioni matriciali e vettoriali, nonché la creazione degli alberi per oggetti di tipo \ac{AABB} e l'intersezione tra gli stessi.
%
\subsection{\texttt{Eigen3}}
\texttt{Eigen3} è una libreria \texttt{C++} di alto livello di \textit{template headers} per operazioni di algebra lineare, vettoriali, matriciali, trasformazioni geometriche, \textit{solver} numerici e algoritmi correlati.

Questa libreria è implementata usando la tecnica di \textit{template metaprogramming}, che crea degli alberi di espressioni in fase di compilazione e genera un codice personalizzato per valutarli. Utilizzando i modelli di espressione e un modello di costo delle operazioni in virgola mobile, la libreria esegue il proprio srotolamento del loop e vettorializzazione.
%
\subsection{\texttt{Clothoids}}
Questa libreria nasce per il \textit{fitting} dei polinomi di Hermite di tipo $G^1$ e $G^2$ con clotoidi, \textit{spline} di clotoidi, archi circolari e \textit{biarc}. In questo lavoro di tesi la libreria \texttt{Clothoids} è stata usata per sfruttare l'implementazione dell'oggetto albero per oggetti di tipo \ac{AABB}.
%
\section{Utilizzo}
La libreria \texttt{TireGround} è stata pensata per essere semplice da utilizzare. Si vedranno ora i vari passi per utilizzarla in maniera appropriata.
%
\paragraph{Caricare la \textit{mesh}}
Per caricare la superficie stradale, rappresentata dalla \textit{mesh} triangolare contenuta nel file \ac{RDF}, è sufficiente sfruttare il costruttore della classe \texttt{MeshSurface} che prende in \textit{input} l'indirizzo al file.
\begin{pseudoc}
	RDF::MeshSurface Road("./file.rdf");
\end{pseudoc}
%
\paragraph{Creare lo pneumatico}
Per creare lo pneumatico a singolo disco è sufficiente utilizzare il costruttore di \textit{default} della classe \texttt{MagicFormula}.
\begin{pseudoc}
	PatchTire::Tire* SampleTire = new PatchTire::MagicFormula(
		SectionWidth, // Sezione laterale dello pneumatico [mm]
		AspectRatio,  // Aspect ratio percentuale dello pneumatico
		RimDiameter,  // Diametro del cerchio [in]
		Threshold     // Threshold per passare dal modello di contatto ponderato in base all'area di intersezione a quello di Rill 
		);
\end{pseudoc}
Nel caso invece si voglia creare uno pneumatico a più dischi si utilizzerà uno dei costruttori della classe \texttt{MultiDisk}. Per il caso di pneumatico a più dischi con raggio uniforme si avrà:
\begin{pseudoc}
	PatchTire::Tire* SampleTire = new PatchTire::MultiDisk(
		SectionWidth, // Sezione laterale dello pneumatico [mm]
		AspectRatio,  // Aspect ratio percentuale dello pneumatico
		RimDiameter,  // Diametro del cerchio [in]
		PointsNumber, // Numero di punti di campionamento per ogni disco
		DisksNumber,  // Numero di dischi totale
		Threshold     // Threshold per passare dal modello di contatto ponderato in base all'area di intersezione a quello di Rill 
		);
\end{pseudoc}
Nel caso di pneumatico a più dischi con raggio di raccordo sulla spalla si avrà invece:
\begin{pseudoc}
	PatchTire::Tire* SampleTire = new PatchTire::MultiDisk(
		SectionWidth, // Sezione laterale dello pneumatico [mm]
		AspectRatio,  // Aspect ratio percentuale dello pneumatico
		RimDiameter,  // Diametro del cerchio [in]
		SideRadius,   // Raggio di raccordo sulla spalla [mm]
		PointsNumber, // Numero di punti di campionamento per ogni disco
		DisksNumber,  // Numero di dischi totale
		Threshold     // Threshold per passare dal modello di contatto ponderato in base all'area di intersezione a quello di Rill 
		);
\end{pseudoc}
Infine, nel caso si voglia creare uno pneumatico a più dischi con forma personalizzata:
\begin{pseudoc}
	PatchTire::Tire* SampleTire = new PatchTire::MultiDisk(
		SectionWidth, // Sezione laterale dello pneumatico [mm]
		AspectRatio,  // Aspect ratio percentuale dello pneumatico
		RimDiameter,  // Diametro del cerchio [in]
		RadiusVec,    // Vettore dei raggi dei dischi [m]
		PointsNumber, // Numero di punti di campionamento per ogni disco
		Threshold     // Threshold per passare dal modello di contatto ponderato in base all'area di intersezione a quello di Rill 
		);
\end{pseudoc}
%
\paragraph{Orientazione dello pneumatico e valutazione del contatto}
Per orientare lo pneumatico e valutarne il contatto con il manto stradale si utilizzerà il metodo \texttt{setup} della classe \texttt{Tire}.
\begin{pseudoc}
bool Out = TireMD->setup(
	Road,     // Superficie stradale
	TransfMat // Matrice di trasformazione 4x4 per orientare lo pneumatico
	);
\end{pseudoc}
Per estrarre i risultati si andranno dapprima a inizializzazione delle variabili reali o vettoriali come segue.
\begin{pseudoc}
	// Inizializzazione delle variabili
	PatchTire::vec3 N;
	PatchTire::vec3 P;
	PatchTire::real_type Friction;
	PatchTire::real_type Rho;
	PatchTire::real_type RhoDot;
	PatchTire::real_type RelativeCamber;
	PatchTire::real_type Friction;
	PatchTire::real_type Area;
	PatchTire::real_type Volume;
	PatchTire::real_type RelativeCamber;
	
	// Estrazione della dimensione appropriata della struttura dati
	PatchTire::int_type size = TireSD->getDisksNumber();
	
	// Inizializzazione dei vettori con dimensione appropriata
	PatchTire::row_vec3 NVec(size);
	PatchTire::row_vec3 PVec(size);
	PatchTire::row_vecN FrictionVec(size);
	PatchTire::row_vecN RhoVec(size);
	PatchTire::row_vecN RhoDotVec(size);
	PatchTire::row_vecN RelativeCamberVec(size);
	PatchTire::row_vecN FrictionVec(size);
	PatchTire::row_vecN AreaVec(size);
	PatchTire::row_vecN VolumeVec(size);
	PatchTire::row_vecN RelativeCamberVec(size);
\end{pseudoc}
Successivamente verranno modificate dai metodi della classe \texttt{Tire} come:
\begin{pseudoc}
	// Estrazione dei dati
	SampleTire->getNormal(N);
	SampleTire->getPoint(P);
	SampleTire->getFriction(Friction);
	SampleTire->getRho(Rho);
	SampleTire->getRhoDot(PreviousRho,TimeStep,RhoDot);
	SampleTire->getRelativeCamber(RelativeCamber);
	SampleTire->getArea(Area);
	SampleTire->getVolume(Volume);
	SampleTire->getRelativeCamber(RelativeCamber)
	
	// Estrazione dei dati in vettori
	SampleTire->getNormal(NVec);
	SampleTire->getPoint(PVec);
	SampleTire->getFriction(FrictionVec);
	SampleTire->getRho(RhoVec);
	SampleTire->getRhoDot(PreviousRho,TimeStep,RhoDotVec);
	SampleTire->getRelativeCamber(RelativeCamberVec);
	SampleTire->getArea(AreaVec);
	SampleTire->getVolume(VolumeVec);
	SampleTire->getRelativeCamber(RelativeCamberVec)
\end{pseudoc}
%
\paragraph{Casi particolari}
Nel caso in cui la variabile booleana in \textit{output} dal metodo \texttt{setup} precedentemente chiamato sia falsa, si prospettano due casi.
%
\paragraph{Pneumatico fuori \textit{mesh}} Se, oltre alla condizione sulla variabile booleana in \textit{output}, la lista di triangoli intersecati dall'ombra dello pneumatico è vuota. Per effettuare questo test si può intersecare l'albero della \textit{mesh} con l'albero a una foglia di dell'ombra dello pneumatico.
\begin{pseudoc}
	bool List = Road.intersectAABBtree(SampleTire.getAABBtree());
\end{pseudoc}
Se la variabile booleana in \textit{output} dal metodo \texttt{intersectAABBtree} è falsa allora la lista è vuota e lo pneumatico sarà quindi considerato fuori dalla superficie stradale descritta nel file \ac{RDF}.
%
\paragraph{Pneumatico in volo}
Se, oltre alla condizione sulla variabile booleana in \textit{output}, la lista di triangoli intersecati dall'ombra dello pneumatico non è vuota. Per effettuare questo test si può intersecare l'albero della \textit{mesh} con l'albero a una foglia di dell'ombra dello pneumatico. I parametri d'intersezione vanno settati dall'esterno della libreria \texttt{TireGround} a seconda della morfologia del terreno fuori dalla \textit{mesh}.
\begin{pseudoc}
	bool List = Road.intersectAABBtree(SampleTire.getAABBtree());
\end{pseudoc}
Se la variabile booleana in \textit{output} dal metodo \texttt{intersectAABBtree} è vera allora la lista non è vuota e lo pneumatico sarà quindi considerato in volo sopra la superficie stradale descritta nel file \ac{RDF}. In questo caso i parametri d'intersezione vanno settati come intersezione nulla.
%
\section{Prestazioni}
Testando il modello sul simulatore è stato possibile valutare la sua complessità computazionale. Le specifiche tecniche di questo simulatore sono:
\begin{itemize}
	\item memoria RAM: 32 GB;
	\item processore: Intel Xeon(R) 3.40 GHz (16 \textit{cores});
	\item scheda grafica: Nvidia GeForce GTX 680.
\end{itemize}

I tempi rilevati sono...

