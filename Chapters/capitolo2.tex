\chapter{Lo pneumatico}
\label{Pneumatico}
%
\section{Introduzione}
Gli pneumatici sono probabilmente i componenti più complessi di un'auto in quanto combinano decine di componenti che devono essere formati, assemblati e combinati assieme. Il successo del prodotto finale dipende dalla loro capacità di fondere tutti i componenti separati in un prodotto dal materiale coeso che soddisfa le esigenze del conducente \cite{rill}. Essi sono caratterizzati da un comportamento altamente non lineare con una forte dipendenza da diversi fattori costruttivi e ambientali.
%
\section{Geometria}
Quando si fa riferimento ai dati puramente geometrici, viene utilizzata una forma abbreviata della notazione completa prevista dall'ente di normazione \ac{ETRTO}. Assumendo di avere un pneumatico generico la notazione che identificherà la geometria sarà del tipo $a$ / $b$ R $c$. Dove:
\begin{itemize}
	\item $a$ rappresenta larghezza nominale dello pneumatico nel punto più largo;
	\item $b$ rappresenta percentuale dell'altezza della spalla dello pneumatico in relazione alla larghezza dello stesso;
	\item $c$ rappresenta il diametro dei cerchi ai quali lo pneumatico si adatta.
\end{itemize}
Si prenda come esempio la seguente denominazione \ac{ETRTO}: 195/55R16. La larghezza nominale dello pneumatico è di circa 195 mm nel punto più largo, l'altezza della spalla corrisponde al 55\% della larghezza $-$ ovvero 107 mm $-$ e il diametro dei cerchi ai quali lo pneumatico si adatta è di 16 pollici. Con questa notazione è possibile calcolare direttamente il diametro esterno teorico dello pneumatico tramite una delle seguenti formule:
%
\begin{equation}
\phi_e = \frac{2ab}{25.4}+c \quad \text{[in]} \qquad
\end{equation}
\begin{equation}
\phi_e = 2ab+25.4c \quad \text{[mm]}
\end{equation}
%
\noindent
Riprendendo l'esempio usato sopra, il diametro esterno risulterà dunque 24.44 in o 621 mm.\\
Meno comunemente usato negli USA e in Europa (ma spesso in Giappone) è una notazione che indica l'intero diametro del pneumatico invece delle proporzioni dell'altezza della spalla laterale, quindi non secondo \ac{ETRTO}. Per fare lo stesso esempio, un cerchio da 16 pollici avrebbe un diametro di 406 mm. L'aggiunta del doppio dell'altezza del pneumatico (2$\times$107 mm) produce un diametro totale di 620 mm. Quindi, un pneumatico 195/55R16 potrebbe in alternativa essere etichettato come 195/620R16. Anche se queste due notazioni sono teoricamente ambigue, in pratica possono essere facilmente distinte perché l'altezza della parete laterale di uno pneumatico automobilistico è in genere molto inferiore alla larghezza. Quindi, quando l'altezza è espressa come percentuale della larghezza, è quasi sempre inferiore al 100\% (e certamente meno del 200\%). Al contrario, i diametri degli pneumatici del veicolo sono sempre superiori a 200 mm. Pertanto, se il secondo numero è superiore a 200, allora è quasi certo che viene utilizzata la notazione giapponese, se è inferiore a 200 allora viene utilizzata la notazione USA/europea.

\begin{figure}[h]
	\centering
	\includegraphics[width=0.7\linewidth]{Figures/tire_measures}
	\caption{Esempio di misure, secondo la notazione ETRTO, riportate sulla spalla dello pneumatico.}
	\label{tiremeasures}
\end{figure}
%
\section{Modellizzazione}
Le forze di contatto tra la superficie stradale e lo pneumatico possono essere descritte da un vettore di forza risultante applicato in un punto specifico dell'impronta di contatto e da una coppia risultante, come illustrato nella \figurename{  \ref{tireforces}}.

\begin{figure}[h!]
	\centering
	\begin{subfigure}{0.4\linewidth}
		$F_x$ \quad forza longitudinale\\
		$F_y$ \quad forza laterale\\
		$F_z$ \quad forza verticale\\
		$T_x$ \quad coppia di sovrasterzo\\
		$T_y$ \quad resistenza al rotolamento\\
		$T_z$ \quad coppia di autoallineamento
	\end{subfigure}
	\begin{subfigure}{0.4\linewidth}
		\centering
		\includegraphics[width=\linewidth]{Figures/tire_forces}
	\end{subfigure}
\caption{Forze e coppie generate dal contatto pneumatico/strada.}
Da: \citeauthor{Rill}, \citetitle{Rill}.
\label{tireforces}
\end{figure}
%
Come componenti cruciali per la movimentazione dei veicoli e il comportamento di guida, le forze degli pneumatici richiedono particolare attenzione soprattutto perché deve essere considerato anche il comportamento non stazionario. Attualmente, è possibile suddividere i modelli di pneumatico in tre gruppi:
\begin{itemize}
	\item modelli matematici;
	\item modelli fisici;
	\item combinazione dei precedenti.
\end{itemize}

\noindent
La prima tipologia di modello tenta di rappresentare le caratteristiche fisiche dello pneumatico attraverso una descrizione puramente matematica. Pertanto, questo tipo di modelli parte da un curve caratteristiche ricavate sperimentalmente e cercano di derivare un comportamento approssimativo dall'interpolazione di un grande insieme di dati. Un esempio ben noto di questo approccio è il \textbf{modello di Pacejka} o \textbf{\textit{Magic Formula}} \cite{hans}. Questo tipo di modellazione è adatta per la simulazione di guida in cui il comportamento di interesse è per lo più la guidabilità del veicolo e le frequenze di uscita sono ben al di sotto delle frequenze di risonanza della cintura dello pneumatico. I modelli fisici o i modelli ad alta frequenza, come i modelli agli elementi finiti, sono in grado di rilevare fenomeni di risonanza a frequenza più elevata. Ciò permette di valutare il comfort di guida di un veicolo. Dal punto di vista del calcolo, i modelli fisici complessi richiedono molto tempo al calcolatore per essere risolti, nonché di molti dati, al contrario dei più veloci modelli matematici, che richiedono un'accurata pre-elaborazione dei dati sperimentali. La terza tipologia di modelli consiste in un'estensione dei modelli matematici attraverso le leggi fisiche al fine di coprire una gamma di frequenza più ampia.

Il modello di pneumatico sviluppato nel modello di veicolo e il tipo di interfaccia di pneumatico/strada presentato da \citeauthor{Larcher} in \cite{Larcher} si basa sulla \textit{Magic Formula} 6.2.
%
\subsection{La \textit{Magic Formula}}
Uno dei modelli di pneumatici più utilizzati è il cosiddetto modello \textit{Magic Formula} sviluppato da \citeauthor{bakker} in \cite{bakker}. Questo modello è stato poi rivisto e l'ultima versione è riportata in \cite{hans}. Il modello \textit{Magic Formula} consiste in una pura descrizione matematica del rapporto input-output del contatto pneumatico/strada. Questa formulazione collega le variabili di forza con lo slip rigido del corpo che vengono trattati nelle sezioni successive. La forma generale della funzione di descrizione può essere scritta come:
%
\begin{equation}
y(x) = D\sin\{C\arctan[B(x + S_h ) - E(B(x + S_h ) - \arctan(B(x + S_h )))]\} + S_v
\end{equation}
%
dove i fattori rappresentano:
\begin{itemize}
	\item $B$ la rigidezza;
	\item $C$ la forma;
	\item $D$ il valore massimo della forza o coppia;
	\item $E$ la curvatura in corrispondenza del valore massimo;
	\item $S_v$ lo spostamento in verticale della curva caratteristica;
	\item $S_h$ lo spostamento in orizzontale della curva caratteristica.
\end{itemize}
e dove $y(x)$ può rappresentare la forza longitudinale $F_x$ , la forza laterale $F_y$ o la coppia di autoallineamento $M_z$, mentre $x$ è la componente di slip corrispondente. In \figurename{ \ref{pacejka}} sono illustrate le curve caratteristiche generiche degli pneumatici derivate con il metodo della \textit{Magic Formula}.\\
Per poter utilizzare la \textit{Magic Formula} è necessario conoscere:
\begin{itemize}
	\item la geometria dello pneumatico;
	\item lo slittamento (o \textit{slip});
	\item la forza verticale applicata allo pneumatico;
	\item la penetrazione in corrispondenza del punto di contatto e la sua derivata nel tempo;
	\item l'inclinazione tra piano strada e sistema di riferimento del centro ruota (angolo di camber relativo).
\end{itemize}
Ed è proprio nell'inclinazione tra piano strada e sistema di riferimento del centro ruota che si porrà una maggiore attenzione in quanto elemento fondamentale per ricavare l'effettivo punto di contatto dell'interazione pneumatico/strada.
%
\begin{figure}[h]
	\centering
	\includegraphics[width=0.58\linewidth]{Figures/pacejka}
	\caption{Curve caratteristiche generiche degli pneumatici derivate con il metodo della \textit{Magic Formula}.}
	Da: \citeauthor{Schramm}, \citetitle{Schramm}.
	\label{pacejka}
\end{figure}
%
\section{Contatto con la superficie stradale}
%
Si analizzaranno ora le quattro metodologie, di complessità cresciente, per ricavare l'inclinazione tra piano strada e sistema di riferimento del centro ruota. Dapprima si utilizzerà un metodo a disco singolo presentato in \cite{Rill}, successivamente si passerà ad un modello a più dischi, così da coprire una superficie stradale maggiore e avere quindi risultati più precisi, soprattutto in prossimità di variazioni repentine del manto stradale.
%
\subsection{Modello di pneumatico a disco singolo}
%
\subsubsection{Contatto di Rill}
La posizione e l'orientamento della ruota in relazione al sistema fissato a terra sono dati dalla terna di riferimento del vettore ruota $RF_{wh}$, che viene calcolata istante per istante risolvendo le equazioni dinamiche del sistema ottenuto nel Capitolo 2 in \cite{Larcher}. Supponendo che il profilo stradale sia rappresentato da una funzione arbitraria a due coordinate spaziali del tipo: 
%
\begin{equation}
z=z(x,y)
\end{equation}
%
su una superficie irregolare, il punto di contatto con il piano locale $P_{PL}$ non può essere calcolato direttamente. Nel metodo a disco singolo presentato in \cite{Rill} da \citeauthor{Rill}, come prima approssimazione si identifica un punto di contatto $P^\star$ come una semplice traslazione del centro ruota $M$:
%
\begin{equation}
P^\star = M-R_0\textbf{\textit{e}}_{zC}
\begin{bmatrix}
x^\star\\
y^\star\\
z^\star
\end{bmatrix}
\end{equation}
%
dove $R_0$ è il raggio dello pneumatico indeformato ed $\textbf{\textit{e}}_{zC}$ è il vettore unitario che definisce l'asse $z_C$ del sistema di riferimento della ruota.

\begin{figure}[h]
	\centering
	\begin{subfigure}{0.45\linewidth}
		\centering
		\includegraphics[width=\linewidth]{Figures/contact_geometry_1}
	\end{subfigure}
	\begin{subfigure}{0.35\linewidth}
		\centering
		\includegraphics[width=\linewidth]{Figures/contact_geometry_2}
	\end{subfigure}
	\caption{Geometria del contatto pneumatico-strada.}
	Da: \citeauthor{Rill}, \citetitle{Rill}.
	\label{contactgeometry}
\end{figure}
%
\noindent
La prima stima del sistema di riferimento del punto di contatto $RF_{P^\star}$ è una terna con origine in $P^\star$ e orientazione degli assi definiti dall'orientazione del sistema di riferimento della ruota. Si noti che l'origine di $RF_{P^\star}$ corrisponde alla proiezione lungo l'asse $z_C$ del sistema di riferimento della ruota.
%
\begin{equation}
RF_{P^\star} = \left[
\begin{array}{ccc|c}
& & & x^\star\\
\multicolumn{3}{c|}{\multirow{3}{*}{\raisebox{20mm}{\scalebox{1.5}{$[R_{RF_{wh}}]$}}}} & y^\star\\
& & & z^\star\\ \hline
0 & ~~0 & 0 & 1
\end{array}\right]
\end{equation}\\
%
Al fine di ottenere una buona approssimazione del piano strada locale in termini di inclinazione longitudinale e laterale, sono stati utilizzati i quattro punti di campionamento $(Q^\star_1, Q^\star_2, Q^\star_3, Q^\star_4)$, rappresentati graficamente in \figurename{ \ref{localtrack}}. I punti di campionamento sono definiti nel sistema di riferimento temporaneo del punto di contatto $RF_{P^\star}$ e lo spostamento longitudinale e laterale sono definiti dall'origine, ovvero dallo stesso $P^\star$. I vettori di spostamento sono definiti come:
%
\begin{equation}
\begin{split}
\textbf{\textit{r}}_{Q^\star_{1,2}} = \pm \Delta x \textbf{\textit{e}}_{xP^\star} = \pm \Delta x \textbf{\textit{e}}_{xC} \\
\textbf{\textit{r}}_{Q^\star_{3,4}} = \pm \Delta y \textbf{\textit{e}}_{yP^\star} = \pm \Delta y \textbf{\textit{e}}_{yC}
\end{split}
\end{equation}
%
e quindi, i quattro punti di campionamento sono:
%
\begin{equation}
\begin{split}
Q^\star_{1,2} = P^\star \pm \textbf{\textit{r}}_{Q^\star_{1,2}} = P^\star \pm \Delta x \textbf{\textit{e}}_{xC} \\
Q^\star_{3,4} = P^\star \pm \textbf{\textit{r}}_{Q^\star_{3,4}} = P^\star \pm \Delta y \textbf{\textit{e}}_{yC}
\end{split}
\end{equation}
%
Al fine di campionare il terreno nel modo più efficace possibile, le distanze di $\Delta x$ e $\Delta y$, dell'equazione precedente, vengono regolate in base al raggio del pneumatico indeformato $R_0$ e alla larghezza del pneumatico $B$. I valori di queste due quantità possono essere trovate in \cite{Rill} e sono $\Delta x = 0.1 R_0$ e $\Delta x = 0.3 B$. Attraverso questa definizione, si può ottenere un comportamento sufficientemente realistico durante la simulazione.

\begin{figure}[h]
	\centering
	\includegraphics[width=0.5\linewidth]{Figures/local_track}
	\caption{Punti campionati nel piano locale della superficie stradale.}
	Da: \citeauthor{Rill}, \citetitle{Rill}.
	\label{localtrack}
\end{figure}
%
\noindent
Ora la componente $z$ in corrispondenza dei quattro punti campione viene valutata attraverso la funzione $z(x,y)$ precedentemente definita. Quindi, aggiornando la terza coordinata dei punti di campionamento $Q^\star_i$, si ottenengono i corrispondenti punti campione $Q_i$ sulla superficie della pista locale. La linea fissata dai punti $Q_1$, $Q_2$ e $Q_3$, $Q_4$, può ora essere utilizzata per definire la normale al piano strada locale (\figurename \ref{localplane}). Pertanto, il vettore normale è definito come:
%
\begin{equation}
\textbf{\textit{e}}_n = \frac{\textbf{\textit{r}}_{Q_1 Q_2} \times \textbf{\textit{r}}_{Q_4 Q_3}}{|\textbf{\textit{r}}_{Q_1 Q_2} \times \textbf{\textit{r}}_{Q_4 Q_3}|}
\label{normale}
\end{equation}
%
Ora, i versori $\textbf{\textit{e}}_x$ ed $\textbf{\textit{e}}_y$, che descrivono l'inclinazione del piano locale nel possono essere ottenuti dalle seguenti equazioni:
%
\begin{equation}
\textbf{\textit{e}}_x = \frac{\textbf{\textit{e}}_{yC}\times\textbf{\textit{e}}_{n}}{|\textbf{\textit{e}}_{yC}\times\textbf{\textit{e}}_{n}|}
\qquad
\textbf{\textit{e}}_{y} = \textbf{\textit{e}}_{n}\times\textbf{\textit{e}}_{x}
\label{terna}
\end{equation}
%
dove sono $\textbf{\textit{r}}_{Q_2 Q_1}$ e $\textbf{\textit{r}}_{Q_4 Q_3}$ sono i vettori che puntano rispettivamente da $Q_1$ a $Q_2$ e da $Q_3$ a $Q_4$. Applicando la \eqref{terna} è ora possibile calcolare i vettori unitari $\textbf{\textit{e}}_{x}$ e $\textbf{\textit{e}}_{y}$ del piano locale di contatto. Per definire univocamente il piano strada, oltre alla normale calcolata in \eqref{normale}, viene utilizzato il punto $P_n$ dato dal valore medio delle tre coordinate spaziali dei quattro punti campione.
%
\begin{equation}
P_n = \frac{1}{4}\begin{bmatrix}
\sum_{i=1}^{4} x_i \\
\sum_{i=1}^{4} y_i \\
\sum_{i=1}^{4} z_i
\end{bmatrix}
\end{equation}
%
È infine necessario ricondursi alle condizioni tali per cui il modello di Pacejka è valido trovando il punto di contatto sul piano strada locale $P_{PL}$ e il punto di contatto sulla circonferenza del disco indeformabile $P_{MF}$ dove effettivamente agiranno le forze ricavate mediante la \textit{Magic Formula}. Si troverà dapprima la componente della normale al piano strada $\textbf{\textit{e}}_{n_{XZ}}$ sul piano in cui giace il singolo disco indeformabile. $P_{MF}$ sarà dunque trovato a partire dal centro ruota $M$, moltiplicando scalarmente il versore $\textbf{\textit{e}}_{n_{XZ}}$ per il raggio del disco indeformabile $R_0$, ovvero:
%
\begin{equation}
P_{MF} = R_0\textbf{\textit{e}}_{n_{XZ}}
\end{equation}
%
Come illustrato in \figurename{ \ref{localplane}}, il punto di contatto sul piano strada locale $P_{PL}$ viene invece calcolato sfuttando un algoritmo di intersezione piano-raggio (che si tratterà nel Capitolo \ref{Geom_Algos}). $P_{PL}$ giacerà dunque sulla proiezione in direzione $-\textbf{\textit{e}}_{n_{XZ}}$ del punto $M$ sulla retta individuata dal punto $P_n$ e normale $\textbf{\textit{e}}_{n_{XZ}}$.

\begin{figure}[h]
	\centering
	\includegraphics[width=0.5\linewidth]{Figures/local_plane}
	\caption{Spostamento del punto di contatto $P_{PL}$ in relazione alla normale $\textbf{\textit{e}}_{n_{XZ}}$.}
	Da: \citeauthor{Rill}, \citetitle{Rill}.
	\label{localplane}
\end{figure}

Infine si può mettere assieme tutte le componenti del piano di riferimento del punto di contatto $P_{MF}$ ottenendo il relativo sistema di riferimento:
%
\begin{equation}
RF_{MF} = \left[
\begin{array}{ccc|c}
& & & x_{P_{MF}}\\
\multirow{3}{*}{\raisebox{20mm}{\scalebox{1.5}{$\left[\textbf{\textit{e}}_x\right]$}}} & \multirow{3}{*}{\raisebox{20mm}{\scalebox{1.5}{$\left[\textbf{\textit{e}}_y\right]$}}} & \multirow{3}{*}{\raisebox{20mm}{\scalebox{1.5}{$\left[\textbf{\textit{e}}_z\right]$}}} & y_{P_{MF}}\\
& & & z_{P_{MF}}\\ \hline
0 & 0 & 0 & 1
\end{array}\right]
\end{equation}\\
Attraverso questo approccio, la normale del piano strada locale $\textbf{\textit{e}}_{n}$ insieme al punto di contatto sul piano strada locale $P_{PL}$ e al punto di contatto sulla circonferenza del disco indeformabile $P_{MF}$, sono in grado di rappresentare l'irregolarità della strada in modo soddisfacente ma comunque approssimativo, infatti, bordi taglienti o discontinuità del manto stradale saranno smussate da questo approccio.

Nel caso specifico di questo lavoro la superficie stradale non è rappresentata da una funzione del tipo $z(x,y)$ ma bensì da una serie di triangoli. Questo comporta l'impossibilità di valutare la terza coordinata dei punti di campionamento $Q_i^\star$. Per sopperire a questo problema si utilizzerà l'algoritmo per l'intersezione tra raggio e triangolo presentato nel Capitolo \ref{Geom_Algos}. Si definirano dunque i punti di origine dei raggi direttamente nel sistema di riferimento della ruota $RF_{wh}$ come:
\begin{equation}
\begin{split}
Q^\star_{1,2} = M \pm \textbf{\textit{r}}_{Q^\star_{1,2}} = P^\star \pm \Delta x \textbf{\textit{e}}_{xC} \\
Q^\star_{3,4} = M \pm \textbf{\textit{r}}_{Q^\star_{3,4}} = P^\star \pm \Delta y \textbf{\textit{e}}_{yC}
\end{split}
\end{equation}
dai quali partiranno i raggi con direzione $-z_C$ che intersecheranno la \textit{mesh} nei punti $(Q_1, Q_2, Q_3, Q_4)$.
%
\subsubsection{Contatto ponderato in base all'area}
Alternativamente a quello appena visto, si può utilizzare un modello di contatto ponderato in base all'area di intersezione. In altre, parole si andrà a valutare triangolo per triangolo l'intersezione con il disco indeformabile. Prima di tutto di intersecherà il triangolo nello spazio con il piano in cui giace il disco trovando dunque un segmento. Successivamente si valuterà l'intersezione di questo segmento con il disco, valutando l'area tra il segmento stesso e il semicerchio inferiore del disco. Attraverso questa area si potrà pesare la normale alla faccia del triangolo considerato e quindi effettuare una media ponderata con tutti gli atri triangoli che insersecano il disco.

% DISEGNARE AREA TRA SEGMENTO  E DISCO

Questo metodo è ovviamente utilizzabile solo nel caso di strada rappresentata tramite \textit{mesh} triangolare. A differenza dal modello di \cite{Rill}, permette di non approssimare la superficie stradale mediante soli quattro punti ma invece di avere una rappresentazione che sfrutta tutti i dati messi a disposizione dalla discretizzazione del manto stradale.
%
\subsection{Modello di pneumatico a più dischi d'instersezione}
%
Nel modello a più dischi, lo pneumatico sarà rappresentato da più dischi rigidi indeformabili disposti uniformemente lungo la sezione dello stesso. Essi potranno avere raggio uguale o diverso l'uno dall'altro, in modo da rappresentare una forma specifica dello pneumatico.

% DISEGNARE DISCHI COSTRUTTORI

Anche se questo modello di pneumatico è costituito da più dischi, il punto di contatto $P_{MF}$ utilizzato per valutare la formula di Pacejka verrà comunque considerato nel disco "virtuale" giacente sul piano mediano dello pneumatico. Equivalentemente, anche il punto di contatto con la \textit{mesh} $P_{PL}$ verrà considerato nel mediesimo piano.
%
\subsubsection{Contatto tramite campionamento}

%
\subsubsection{Contatto ponderato in base all'area d'instersezione}
%

