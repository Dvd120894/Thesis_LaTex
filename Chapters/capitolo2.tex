\chapter{La Superficie Stradale}
\label{rdf}
%
\section{Introduzione}
Oltre allo pnumatico, la superfice stradale rappresenta il secondo importante elemento che definisce il contatto. Perchè una superficie stradale possa essere facilmente utilizzata da un calcolatore deve essere prima discretizzata. La discretizzazione in questo caso avviene mediante la rappresentazione della superficie stessa in una \textit{mesh} triangolare. La \textit{mesh}, è contenuta in un file formato \ac{RDF}, che contiene le posizione $(x,y,z)$ di ogni vertice e i numeri di identificazione per ognuno dei tre vertici del triangolo, per ogni trangolo.

È importante notare che la discretizzazione del manto stradale è un passaggio molto importante in quando, se campionatonato troppo grossolanamente potrebbe influire negativamente sui risultati dei calcoli per l'estrazione del piano strada locale. In altre parole, una semplificazione troppo spinta, potrebbe causare degli errori tali da incorrere in risultati troppo approssimativi e non rispecchianti la realtà. Al contrario, una \textit{mesh} troppo fitta, complica inutilmente i calcoli, dilatando sensibilmente i tempi di esecuzione. È bene quindi discretizzare più densamente in maniera oculata e solo dove occore realmente, ovvero in prossimità di cordoli, marciapiedi o qualsiasi tipo di ostacolo che potrebbe influire sulle performance della vettura. 
%
\section{Il formato RDF}
Sfortunatamente, non esistono standard universalmente riconosciuti per il formato RDF. In linea di massima le superfici stradali sono definite nei \textit{road data file} (*.rdf). Questa tipologia di file è composto da varie sezioni, indicate da parentesi quadre.

\begin{lstlisting}[language = C++, basicstyle=\ttfamily\small, basewidth=0.55em]
	! Comments section
	
	[UNITS]
	LENGTH = 'meter'
	FORCE = 'newton'
	ANGLE = 'degree'
	MASS = 'kg'
	TIME = 'sec'
	
	[MODEL]
	ROAD\_TYPE = '...'
	
	[PARAMETERS]
	...
\end{lstlisting}

Nella sezione \texttt{[UNITS]}, vengono impostate le unità utilizzate nel file di dati stradali. La sezione \texttt{[MODEL]} viene invece utilizzata per specificare il tipo di strada, del tipo:
\begin{itemize}
	\item \texttt{ROAD\_TYPE = 'flat'}: come indica già il nome, si tratta di una superficie stradale piana.
	\item \texttt{ROAD\_TYPE = 'plank'}: dove questa strada è composta da un singolo scalino o dosso orientato perpendicolarmente o obliquo rispetto all'asse $X$, con o senza bordi smussati.
	\item \texttt{ROAD\_TYPE = 'poly\_line'}: ovvero l'altezza della strada è in funzione della distanza percorsa.
	\item \texttt{ROAD\_TYPE = 'sine'}: dove la superficie stradale è costituita da una o più onde sinusoidali con lunghezza d'onda costante.
\end{itemize}
La sezione \texttt{[PARAMETERS]} contiene parametri generali e parametri specifici del tipo di superficie stradale.

I parametri per ogni tipologia di superficie stradale sono elencati di seguito:
\begin{itemize}
	\item Generali:
	\begin{itemize}
		\item \texttt{MU}: è il fattore di correzione dell'attrito stradale (non il valore dell'attrito stesso), da moltiplicare con i fattori di ridimensionamento LMU del modello di pneumatico.\\
		Impostazione predefinita: \texttt{MU = 1.0}.
		\item \texttt{OFFSET}: è l'offset verticale del terreno rispetto al sistema di riferimento inerziale.
		\item \texttt{ROTATION\_ANGLE\_XY\_PLANE}: è l'angolo di rotazione del piano $XY$ attorno all'asse $Z$ della strada, ovvero la definizione dell'asse $X$ positivo della strada rispetto al sistema di riferimento inerziale.
	\end{itemize}
	\item Strada con scalino:
	\begin{itemize}
		\item \texttt{HEIGHT}: altezza dello scalino.
		\item \texttt{START}: distanza lungo l'asse $X$ della strada all'inizio dello scalino.
		\item \texttt{LENGTH}: lunghezza dello scalino (escluso lo smusso) lungo l'asse $X$ della strada.
		\item \texttt{BEVEL\_EDGE\_LENGTH}: lunghezza del bordo smussato a $45°$ dello scalino.
		\item \texttt{DIRECTION}: rotazione dello scalino attorno all'asse $Z$, rispetto all'asse $Y$ della strada.\\
		Se lo scalino è posizionato trasversalmente, \texttt{DIRECTION = 0}. Se lo scalino è posto lungo l'asse $X$, \texttt{DIRECTION = 90}.
	\end{itemize}
	\item Polilinea:\\
	Il blocco \texttt{[PARAMETERS]} deve avere un sottoblocco chiamato \texttt{(XZ\_DATA)} e costituito da tre colonne di dati numerici:
	\begin{itemize}
		\item La colonna 1 è un insieme di valori $X$ in ordine crescente.
		\item Le colonne 2 e 3 sono insiemi di rispettivi valori $Z$ per la traccia sinistra e destra.
	\end{itemize}
	Esempio:
	\begin{lstlisting}[language = C++, basicstyle=\ttfamily\small, basewidth=0.55em]
	[PARAMETERS]
	MU = 1.0$ peak friction scaling coefficient
	OFFSET = 0.0 $ vertical offset of the ground wrt inertial frame
	ROTATION_ANGLE_XY_PLANE = 0.0 $ definition of the positive X-axis of the road wrt inertial frame
	
	$X_road	Z_left	Z_right
	(XZ_DATA)
	-1.0e04	0	0
	0.0500	0	0
	0.1000	0	0
	0.1500	0	0
	...  ... ...
	\end{lstlisting}
	\item Sinusoide:\\
	La strada a superficie sinusoidale è implementata come:
	\begin{equation}
	z(x)=\frac{H}{2}\left( 1 - \cos \left( \frac{2\pi \cdot (x-x_i)}{L} \right)   \right) 
	\end{equation}
	dove	
	\begin{itemize}
	 	\item $z$: coordinata verticale della strada;
	 	\item $H$: altezza;
	 	\item $x$: posizione attuale;
	 	\item $x_i$: inizio dell'onda sinusoidale;
	 	\item $L$: semi-periodo dell'onda sinusoidale.
	\end{itemize}
	I parametri sono:	
	\begin{itemize}
		\item \texttt{HEIGHT}: altezza dell'onda sinusoidale.
		\item \texttt{START}: distanza lungo l'asse $X$ della strada all'inizio dell'onda sinusoidale.
		\item \texttt{LENGTH}: lunghezza dell'onda sinusoidale lungo l'asse $X$ della strada.
		\item \texttt{DIRECTION}: rotazione dell'onda sinusoidale attorno all'asse $Z$, rispetto all'asse $Y$ della strada.\\
		Se l'onda sinusoidale è posizionata trasversalmente, \texttt{DIRECTION = 0}. Se l'onda sinusoidale è posta lungo l'asse $X$, \texttt{DIRECTION = 90}.
	\end{itemize}
\end{itemize}
Sfortunatamente, queste informazioni permettono di costruire strade troppo semplici che non rispecchiano la realtà. È quindi necessario inserire i risultati della discretizzazione della superficie stradale sopra citati.

Per descrivere una superficie stradale composta da una \textit{mesh} di triangoli si userà la seguente formattazione del file.
\begin{itemize}
	\item \texttt{[NODES]}: presenti nella prima sezione e dove vengono descritti sotto forma di una quartina $(id,x,y,z)$ data dal numero di identificazione e dalle coordinate nello spazio.
	\item \texttt{[ELEMENTS]}: presenti nella seconda sezione e dove vengono descritti sotto forma di una quartina $(n_1,n_2,n_3,\mu)$ data dal numero di identificazione dei tre vertici componenti $i$-esimo triangolo e dal coefficente di attrito presente nella faccia.
\end{itemize}
Esempio:
\begin{lstlisting}[language = C++, basicstyle=\ttfamily\small, basewidth=0.55em]
[NODES]
{ id x_coord y_coord z_coord }
0 2.64637 35.8522 -1.59419e-005 
1 4.54089 33.7705 -1.60766e-005 
2 4.52126 35.8761 -1.62482e-005 
3 2.66601 33.7456 -1.57714e-005 
4 0.771484 35.8282 -1.56367e-005 
5 0.791126 33.7206 -1.5465e-005
... ... ... ...
[ELEMENTS]
{ n1 n2 n3 mu }
1 2 3 1.0 
2 1 4 1.0 
5 4 1 1.0 
... ... ... ...
\end{lstlisting}
%
\section{Il \textit{Parser}}